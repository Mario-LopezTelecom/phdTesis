\chapter{Final remarks}\label{Conclusion_chap}

\section{Summary of conclusions} % a.k.a. ¿Que hemos hecho?

We have addressed the shortcomings of existing Dynamic Spectrum Access (DSA) mechanisms, especially in the protection and incentives for licensed users. Our proposals bring novel looks to DSA's open challenges. 

%OSA
\textbf{OSA}. Under the framework of Opportunistic Spectrum Access (OSA), we have specifically explored two areas: the sensing-throughput trade-off in hardware-constrained radios (chapters \ref{BD_chap} and \ref{SPEC_MAN_chap}) and PU spectrum management for coexistence-friendly SU access (chapters \ref{BR_chap} and \ref{SPEC_MAN_chap}). 

We have improved the SU's periodic sensing approach by adding a novel background detection mechanism (chapter \ref{BD_chap}), which achieves better SU throughput while reducing collision probability with the PU. Our mechanism is simple to implement and robust against estimation inaccuracies. In chapter \ref{SPEC_MAN_chap}, we have proposed an MDP-based sequential sensing policy that takes into account the loss of reliability of a sensing result over time. Such policy improves SU throughput and reduces collision probability with the PUs. Many previous works have considered that channel states are static during the sensing phase, which is a strong assumption in many real scenarios. 

We have shown that, through coexistence-friendly spectrum allocation, a primary operator could significantly improve SU's throughput while improving or not harming its own. More specifically, the primary operator could opt for either reserving part of its spectrum for SU access (chapter \ref{BR_chap}), or allocating channels to primary users in a sequential way (chapter \ref{SPEC_MAN_chap}). 
Therefore, a primary operator would use this simple coexistence measure in its self-interest, even though the initial motivation of cognitive radio (especially under OSA) was SU transparent access. 

Finally, we showed an application of OSA beyond the abstract terms \enquote{PU} and \enquote{SU}: a mechanism for an operator to reuse its legacy licensed and underused cellular network (PUs) by newer cognitive terminals (SUs) (chapter \ref{RSM_chap}). This mechanism is based on a novel manner of simultaneously exploiting temporal and spatial spectrum opportunities, improving the system's capacity while protecting the legacy PUs. The proposed mechanism comprises an optimal learning algorithm to find the solution of an stochastic optimization problem. As additional benefits, this approach is highly adaptive and model-free.

%Trading
\textbf{Automated spectrum trading}. Regarding automated spectrum trading, we have proposed two mechanisms providing different types of incentives to PUs: economic (chapter \ref{Sarnoff_chap}) and relay services (chapter \ref{MAB_CSSA_chap}). The mechanism described in chapter \ref{Sarnoff_chap}, which is especially well-suited for lightly-loaded primary networks, aims to balance the blocking probability of PUs with the revenue obtained from leasing channels to SUs. The work developed in chapter \ref{MAB_CSSA_chap} is more appropriate for heavy-loaded primary networks. In that scenario, the SUs help to increase the data rate of the PUs in exchange for spectrum opportunities and as a consequence, create more spectrum opportunities. This later mechanism does not assume any knowledge about the SUs and thus, incorporates a learning mechanism of payoff-maximizing actions. Both proposed mechanisms are simple to implement and to extend to more complex scenarios, do not make unrealistic assumptions and offer a dramatic improvement over baseline approaches. %TO DO: mejorar estas conclusiones, son pobres.

In addition, we have elaborated a comprehensive taxonomy in the topic providing evidences of one of our main points in this thesis: that the research community was not addressing the incentives problem of PUs correctly. As we stated in the introduction of the thesis, researchers have been focused on maximizing the profit of PUs, and as a consequence, building increasingly complex models that are not suitable for real-time operation.

\section{So what?} % Implicaciones de cara al objetivo principal (+ nuevas aplicaciones: Legacy Networks, HetNets)
We believe this thesis represents a significant push in making DSA a more attractive mechanism for the wireless industry to meet the explosion in mobile data demand. 

Such attractiveness is relevant given the reluctance to implement DSA of incumbent mobile operators and other spectrum owners like TV broadcasters, despite their claims of a \enquote{spectrum crisis} \cite{Chen2012b}. To date, both regulators and operators have only taken baby steps towards exploiting the full potential of DSA, under any of its possible frameworks \cite{Nuechterlein2013, ref:Kelly2012}.

As we stated in the introduction, the wireless industry main concerns regarding DSA are interferences and the absence of incentives for them to adopt it. They do not only argue that the risk of interference is harmful to their activities (when they act as primary users), but also that it discourages investments (when they act as secondary users, for example, in the federal band between 1755-1850 Mhz), since they cannot predict how much they will be able to exploit a given shared band \cite{CTIA2011}.  Another surprising objection is, as Chris Guttman-McCabe (vice president of regulatory affairs at CTIA - The Wireless Association) puts it, that \enquote{we are not aware that [cognitive] technology exists in a really truly commercially viable or scalable format [for advanced spectrum sharing]} \cite{Chen2012}. The principal interest of the wireless industry is for the governments to clear more spectrum for their exclusive use \cite{Roche2015,Chen2012}. 

Most of the work during the last 10 years in DSA (1367 works listed in Thompson Reuters' Web of Science) including this thesis, as well as important figures such as the President's Council of Advisors on Science and Technology (PCAST) \cite{AdvisorsonScience}, or Martin Cooper, the inventor of the cellphone \cite{Chen2012b,Br2012}, counterargument the industry's objections. Martin Cooper asserts, however, that the true reason behind the industry's opposition to sharing is that \enquote{licensing spectrum is a zero-sum game. When a company gets the license for a band of radio waves, it has the exclusive rights to use it. Once a company owns it, competitors can't have it}.

Regulators also have some objections to the widespread use of DSA, namely political pressures from the industry (\textit{e.g.} the broadcasting lobby with regard to TV white spaces \cite[p. 103]{Nuechterlein2013}), and the loss of an important source of revenue: spectrum auctions \cite{Baker2015}.

In any case, the work developed in this thesis is also relevant to other  approaches enjoying more support from the industry, such as DSA for coexistence with legacy networks, as our proposal in chapter \ref{RSM_chap}, and for coexistence with small cells (HetNets) enriched with cognitive capabilities, as well as \enquote{voluntary} spectrum trading. 

\section{Future Work}
The conflict of interest between the industry and social welfare has no trivial solution and may even depend on political views on the matter. 

It is worth noting that the wireless industry does not roundly refuse DSA. In fact, they claim to be already using it within their networks and have sharing agreements \textit{e.g.} for Amazon's Kindle or for Garmin's GPS navigator \cite{CTIA2011}. What they refuse is enforced, unpredictable sharing of spectrum. Thus, regarding the viability of adoption, it is a safer approach for the research community to further study those approaches approved by the incumbents:
\begin{itemize}
\item Authorized / Licensed Shared Access  
\item HetNets
\item Automated spectrum trading
\item Millimeter wave spectrum
\end{itemize}

\textit{Authorized or Licensed Shared Access (LSA)} is a DSA initiative proposed by the industry. For SUs to transmit in a spectrum band operating under LSA, they must obtain a license from the PU and follow a set of negotiated conditions. LSA relies in geo-location databases, as an alternative to spectrum sensing, preferred by incumbents and regulators alike 3 % Podria incluir mas info. de los retos que implica este acceso. 
We could also include in this point DSA within the operator's networks. As we modeled in chapter \ref{RSM_chap}, an operator may be willing to access its legacy network spectrum resources with newer terminals, without incurring in the cost of upgrading its old equipment to support coexistence. The challenges to face in this scenario are similar to those in plain OSA, although protection to PUs is likely to lose importance in favor of overall performance, seeking Pareto optimal points of operation (as we did in chapter \ref{SPEC_MAN_chap}). 

As we introduced in chapter \ref{SPEC_MAN_chap}, \textit{HetNets} also imply two-layer networks, in this case an overlay of small cells over a macro-cell network, usually with no structured deployment. Due to their unplanned nature and density, it is convenient to equip these small cells with cognitive abilities to avoid interference with macro cell transmissions. As in the previous approach, a substantial difference with classical interweave OSA is that all the HetNet users are PUs and thus (users of the same operator), and QoS must be guaranteed for both the macro and the small cell users. 

With regard to \textit{automated spectrum trading}, summarizing what we explained in the conclusions of chapter \ref{Survey_chap}, the main focus should be on real-time adaptation instead of optimality, applications in practical network environments, alternatives to economic trades such as Cooperative Spectrum Sharing, protection against market failures (collusions, lack of rationality, etc.), and more complex analytic studies (cooperative game theory, stochastic games, learning mechanisms).

\textit{Millimeter wave spectrum} is strongly in line with the industry's desires of more spectrum. The cost and power consumption of its required equipment has decreased and previous propagation issues are now seen as surmountable \cite{ref:Andrews2014}. They key challenge it brings is \enquote{beamforming}, networking through narrow beams. Nevertheless, it also unlocks a new dimension for spectrum sharing. Combined with small cells, millimeter wave spectrum may become the future \enquote{beachfront spectrum}\footnote{Beachfront spectrum is how the band between 225 Mhz and 3,7 Ghz is informally called, which is highly valuated because of its propagation characteristics \cite{AdvisorsonScience}.}

Although a deep study on regulation issues is out of the scope of this thesis, it is impossible to neglect its influence in the successful implementation of DSA. Policymakers should be focused on: 
\begin{itemize}
\item Spectrum sharing policy enforcement and monitoring
\item Standardization and design of general interference guidelines
\item Design of incentive mechanisms for spectrum owners
\item Micro-economic market studies 
\end{itemize} 

Policymakers should \textit{enforce sharing in some cases}, despite the warnings of the wireless industry. This point should not be seen as contrary to our motivation or to the previous discussion, but more like a complementary and simultaneous approach, as we indicated in chapter \ref{BR_chap}. There is enough evidence of the benefits of  spectrum sharing for the social welfare, and reallocation is not always convenient, as in the federal band. Contrary to the claims of the industry, this may force them to invest in new technologies to improve coexistence. Along the same line, regulators could enforce build-out milestones and roll-out obligations. In addition, complex disputes over interferences are expected, thus an intense \textit{effort in monitoring} will be needed. Even advocates of the free-market should take into account that the transition from the current situation may need the regulator intervention to avoid harm to social welfare \cite{ref:Yoon2012}. The degree of interventionism should be carefully balanced with its administrative burden, so as not to lose dynamism. 

Regulators could also apply \text{incentive mechanisms for spectrum owners} to give up part of their spectrum. This is a controversial point, since most of them got access to spectrum for free and giving them a compensation could be considered by the society as unfair. Nevertheless, taking decisions based on historical fairness considerations instead on the current social welfare interests is a known fallacy called the \enquote{sunk-cost fallacy} \cite[p. 111]{Nuechterlein2013}. 

With respect to \textit{standardization and interference guidelines}, their positive impact is twofold: first, they will help streamline the administrative processes, and second, they will help remove uncertainty about how much spectrum each entity gets to use, which is one of the strongest objections of the wireless industry incumbents to DSA. Standardization is especially critical in Europe, where each country has pursued its own spectrum policies.
 
Finally, as pointed out in the Report on Radio Spectrum Competition issues, by the European Radio Spectrum Policy Group \cite{ref:RSPG2009}, there is a need for more complex \textit{micro-economic market studies}, such as the long-term impact of DSA and possible anticompetitive outcomes. Also quoting from the same report, \enquote{to date there is limited practical experience}: more insight in real-life implementation of DSA is necessary, although there have been important steps in that direction recently \cite{Matinmikko2013}. 

%Finally, it is also worth mentioning that pursuing an increase in data capacity should be carefully balanced with other objectives such as increasing security, low-latency, and reliability, which would unlock advanced uses of wireless communications. \emph{Security} is a critical factor in next generation networks, given the billions of devices that are expected to join what is called \enquote{the Internet of Things}, some of them affecting the life of millions of people (\textit{e.g.} driverless cars, power grids, medical equipment), coupled with a daily dose of news alerting of important security breaches. DSA also brings its own security threats such as Primary User Emulation attacks \cite{Chen2008}. \emph{Low-latency} applications are grouped around the umbrella term \enquote{the tactile Internet} \cite{Fettweis2014} and range from recreational uses (as in online gaming), to society-transforming ones (e.g. exoskeletons for disabled people). Lastly, also connected with the Internet of Things, and because of increasing levels of complexity in the network architecture, self-healing capabilities \cite{Zhang2013_self} will become key to enhance the reliability of communications as they scale up. This is also relevant in the topic of our thesis, since the wireless industry claims DSA has low scalability and convergence. 


 




