\chapter{Final remarks}\label{Conclusion_chap}

\section{Summary of conclusions} % a.k.a. ¿Que hemos hecho?

We have addressed the shortcomings of existing Dynamic Spectrum Access (DSA) mechanisms, especially in the protection and incentives of licensed users. Our proposals bring novel looks to DSA's open challenges. 

%OSA
\textbf{OSA}. Under the framework of Opportunistic Spectrum Access (OSA), we have specifically explored two areas: the sensing-throughput trade-off in hardware-constrained radios (chapters \ref{BD_chap} and \ref{SPEC_MAN_chap}) and PU spectrum management for coexistence-friendly SU access (chapters \ref{BR_chap} and \ref{SPEC_MAN_chap}). 

We have improved the SU's periodic sensing approach by adding a novel background detection mechanism (chapter \ref{BD_chap}). It achieves better SU throughput while reducing collision probability with the PU. Our mechanism is simple to implement and robust against estimation inaccuracies. In chapter \ref{SPEC_MAN_chap}, we have proposed an MDP-based sequential sensing policy that takes into account the loss of reliability of a sensing result over time. Such policy improves SU throughput and reduces collision probability with the PUs. Many previous works have considered that channel states are static during the sensing phase, which is a strong assumption in most real scenarios. 

We have proven that, through coexistence-friendly spectrum allocation, a primary operator could significantly improve SU's throughput while improving or not harming its own. More specifically, the primary operator could opt for reserving part of its spectrum for SU access (chapter \ref{BR_chap}) or for allocating sequential channels to primary users(chapter \ref{SPEC_MAN_chap}. 
This is a significant result given that the initial motivation of cognitive radio (especially under OSA) was SUs transparent access. 
It is also non-intuitive, since bandwidth reservation is convenient for the primary operator despite its inability to exploit the channels with best instantaneous gains. 

Finally, we showed an application of OSA beyond the abstract terms \enquote{PU} and \enquote{SU}: a mechanism for an operator to reuse its legacy licensed and underused cellular network (PUs) by newer cognitive terminals (SUs) (chapter \ref{RSM_chap}). Such mechanism is based on a novel manner of exploiting temporal and spatial spectrum opportunities, significantly improving the system's capacity while protecting the legacy PUs. It is a promising approach since it already offers a good protection to PUs without enforcing performance constraints in the formulation (which is left for future work).

%Trading
\textbf{Automated spectrum trading}. Regarding automated spectrum trading, we have proposed two mechanisms providing different types of incentives to PUs: economic (chapter \ref{Sarnoff_chap}) and relay services (chapter \ref{MAB_CSSA_chap}). The mechanism described in chapter \ref{Sarnoff_chap} is especially well-suited for scenarios in which the PUs are light-loaded. The mechanism aims to balance the blocking probability of PUs with the revenue obtained from SU's access. The work developed in chapter \ref{MAB_CSSA_chap} is more appropriate for heavy-loaded primary networks. In that scenario, the SUs help to increase the data rate of the PUs in exchange for PU's spectrum opportunities and as a consequence, create more spectrum opportunities. This later mechanism does not assume any knowledge about the SUs and thus, incorporates a learning mechanism of payoff-maximizing actions. Both are simple to implement and to extend to more complex scenarios, do not make unrealistic assumptions and offer a dramatic improvement over baseline approaches. %TO DO: mejorar estas conclusiones, son pobres.

In addition, we have developed an extensive survey in the topic in which we have given evidences of our main point in this thesis: that the research community was not addressing the incentives problem of PUs correctly. As we stated in the introduction of the thesis, researchers have been focused on maximizing the profit of PUs, and as a consequence, building increasingly complex models that 1) are not suitable for real-time operation, and 2) are not especially concerned with the risk of loss.

\section{So what?} % Implicaciones de cara al objetivo principal (+ nuevas aplicaciones: Legacy Networks, HetNets)
% Hablar del debate sobre mas espectro vs. mas eficiencia.
The ultimate objective of this thesis was to contribute to problem of the explosion in mobile data demands, improving spectrum efficiency by using DSA mechanisms. In order to do turn them into a reality, we believed it is convenient to convince the spectrum owners of using them, instead of relaying on forced regulations to implement them. Therefore, we tackled their two main concerns: protection against interferences and directly providing incentives.

But, as we outlined in the introduction, both regulators and operators are still reticent to DSA and have only taken baby steps towards any of the possible solutions. In conclusion, after the study contained in this thesis, we should be able to answer the following questions: 1) Does DSA improves social welfare? 2) Is it beneficial for the spectrum owners? 3) Did we (both the research community and us)  successfully show its net-benefit to both operators and regulators? 4) If the answer is yes, why its poor adoption rate? If the answer is no, where should we put our efforts now?

We believe that our findings and the vast amount of work during the last 10 years (1367 works listed in Thompson Reuters' Web of Science) represent a clear \enquote{yes} to the first two questions. The answer to question 3, unfortunately, is \enquote{no}, as evidenced by the still expressed fears of regulators and operators \cite{ref:Kelly2012}, however this thesis represents a significant push in the right direction. It is interesting to note that, some of those fears are not well founded. It is unfair to express much concern about, for example, competition issues, spectrum hoarding, and the like, when there is also a non-negligible body of work on the microeconomics of spectrum trading (see chapter \ref{Survey_chap}).

\section{Future Work}
% Decir que esto ya no es tanto un problema tecnologico. 
% Dar tambien directrices tecnicas.
Where should we put our efforts now? It is a matter of directly addressing the comments of regulators and operators:
\begin{itemize}
\item Focus on securing the mechanisms against profit losses
\item Incorporating more real-world issues such as incomplete information, non stationarity, misbehavior of entities, etc., while keeping complexity low, possibly paying the price of sub-optimality of solutions. 
\item Quoting from the Report on Radio Spectrum Competition Issues, by the European Radio Spectrum Policy Group \enquote{to date there is limited practical experience} \cite{ref:RSPG2009}. There is a need of studying the practical issues of implementation, test benches, experimental runs, etc. 
\item Not only the operators are concerned with a possible economic loss, regulator bodies are also worried since static spectrum auctions are an important source of income. 
\item Especially in Europe, where each country has pursued its own spectrum policies, the regulators have to analyze economic research work in DSA, adopt one of the proposed (and well proven) paradigms and design and enforce suitable policies (with more or less intervention from them).
\end{itemize}
This does not mean that research in other sub-areas in not important, for example security in the form of Primary User Emulation (PUE) attacks, long horizon investment strategies, micro-optimization of revenues or load balancing as a primary objective, but the ones we point out are critical to start making DSA a reality. 

Furthermore, work in DSA is applicable not only to its original intention of unlicensed operators, secondary markets and so on. Research in DSA is directly re-usable in recent hot topics such as heterogeneous networks (HetNets) \cite{ref:Damnjanovic2011} and cognitive capabilities in general are central part of next generation networks \cite{ref:Andrews2014}.



