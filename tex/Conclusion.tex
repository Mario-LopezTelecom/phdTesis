\chapter{Final remarks}\label{Conclusion_chap}

\section{Summary of conclusions} % a.k.a. ¿Que hemos hecho?

We have addressed the shortcomings of existing Dynamic Spectrum Access (DSA) mechanisms, especially in the protection and incentives of licensed users. Our proposals bring novel looks to DSA's open challenges. 

%OSA
\textbf{OSA}. Under the framework of Opportunistic Spectrum Access (OSA), we have specifically explored two areas: the sensing-throughput trade-off in hardware-constrained radios (chapters \ref{BD_chap} and \ref{SPEC_MAN_chap}) and PU spectrum management for coexistence-friendly SU access (chapters \ref{BR_chap} and \ref{SPEC_MAN_chap}). 

We have improved the SU's periodic sensing approach by adding a novel background detection mechanism (chapter \ref{BD_chap}). It achieves better SU throughput while reducing collision probability with the PU. Our mechanism is simple to implement and robust against estimation inaccuracies. In chapter \ref{SPEC_MAN_chap}, we have proposed an MDP-based sequential sensing policy that takes into account the loss of reliability of a sensing result over time. Such policy improves SU throughput and reduces collision probability with the PUs. Many previous works have considered that channel states are static during the sensing phase, which is a strong assumption in most real scenarios. 

We have proven that, through coexistence-friendly spectrum allocation, a primary operator could significantly improve SU's throughput while improving or not harming its own. More specifically, the primary operator could opt for reserving part of its spectrum for SU access (chapter \ref{BR_chap}) or for allocating sequential channels to primary users(chapter \ref{SPEC_MAN_chap}. 
This is a significant result given that the initial motivation of cognitive radio (especially under OSA) was SUs transparent access. 
It is also non-intuitive, since bandwidth reservation is convenient for the primary operator despite its inability to exploit the channels with best instantaneous gains. 

Finally, we showed an application of OSA beyond the abstract terms \enquote{PU} and \enquote{SU}: a mechanism for an operator to reuse its legacy licensed and underused cellular network (PUs) by newer cognitive terminals (SUs) (chapter \ref{RSM_chap}). Such mechanism is based on a novel manner of exploiting temporal and spatial spectrum opportunities, significantly improving the system's capacity while protecting the legacy PUs. It is a promising approach since it already offers a good protection to PUs without enforcing performance constraints in the formulation (which is left for future work).

%Trading
\textbf{Automated spectrum trading}. Regarding automated spectrum trading, we have proposed two mechanisms providing different types of incentives to PUs: economic (chapter \ref{Sarnoff_chap}) and relay services (chapter \ref{MAB_CSSA_chap}). The mechanism described in chapter \ref{Sarnoff_chap} is especially well-suited for scenarios in which the PUs are light-loaded. The mechanism aims to balance the blocking probability of PUs with the revenue obtained from SU's access. The work developed in chapter \ref{MAB_CSSA_chap} is more appropriate for heavy-loaded primary networks. In that scenario, the SUs help to increase the data rate of the PUs in exchange for PU's spectrum opportunities and as a consequence, create more spectrum opportunities. This later mechanism does not assume any knowledge about the SUs and thus, incorporates a learning mechanism of payoff-maximizing actions. Both are simple to implement and to extend to more complex scenarios, do not make unrealistic assumptions and offer a dramatic improvement over baseline approaches. %TO DO: mejorar estas conclusiones, son pobres.

In addition, we have developed an extensive survey in the topic in which we have given evidences of our main point in this thesis: that the research community was not addressing the incentives problem of PUs correctly. As we stated in the introduction of the thesis, researchers have been focused on maximizing the profit of PUs, and as a consequence, building increasingly complex models that 1) are not suitable for real-time operation, and 2) are not especially concerned with the risk of loss.

\section{So what?} % Implicaciones de cara al objetivo principal (+ nuevas aplicaciones: Legacy Networks, HetNets)
We believe this thesis represents a significant push in making DSA a more attractive mechanism for the wireless industry to meet the explosion in mobile data demand. 

Such attractiveness is relevant given the reluctance to implement DSA of incumbent mobile operators and other spectrum owners like TV broadcasters, despite their claims of a \enquote{spectrum crisis} \cite{Chen2012b}. To date, both regulators and operators have only taken baby steps towards exploiting the full potential of DSA, under any of its possible frameworks \cite{Nuechterlein2013, Kelly2012}.

As we stated in the introduction, the wireless industry main concerns regarding DSA are protection against interferences and lack of incentives. They do not only argue that the risk of interference is harmful to their activities (when they act as primary users), but also that it discourages investments (when they act as secondary users, for example, in the federal band between 1755-1850 Mhz.), since they cannot predict how much they will be able to exploit a given shared band \cite{CTIA2011}.  Another surprising objection is, as Chris Guttman-McCabe, vice president of regulatory affairs at CTIA-The Wireless Association, puts it, that \enquote{we are not aware that technology [DSA, cognitive radio] exists in a really truly commercially viable or scalable format} \cite{Chen2012}. The principal interest of the wireless industry is for the governments to clear more spectrum for their exclusive use\cite{Roche2015,Chen2012}. 

This thesis, most of the work during the last 10 years in DSA (1367 works listed in Thompson Reuters' Web of Science), and important figures such as the President’s Council of Advisors on Science and Technology (PCAST) \cite{AdvisorsonScience}, or Martin Cooper, the inventor of the cellphone \cite{Chen2012b,Br2012}, counterargument these objections. Cooper even asserts that the true reason behind the industry's opposition to sharing is that \enquote{licensing spectrum is a zero-sum game. When a company gets the license for a band of radio waves, it has the exclusive rights to use it. Once a company owns it, competitors can't have it}.

Regulators also have some objections to the widespread use of DSA, namely political pressures from the industry (\textit{e.g.} the broadcasting lobby with regard to TV white spaces \cite[p. 103]{Nuechterlein2013}), and the loss of an important source of revenue: spectrum auctions \cite{Baker2015}.

In any case, the work developed in this thesis is also relevant to other  approaches enjoying more support from the industry, such as DSA for coexistence with legacy networks, as our proposal in chapter \ref{RSM_chap}, and for coexistence with small cells (HetNets) enriched with cognitive capabilities, as well as \enquote{voluntary} spectrum trading. 

\section{Future Work}
The conflict of interest between the industry and social welfare has no trivial solution and may even depend on political views on the matter. 

It is worth noting that the wireless industry does not oppose to everything related to DSA. In fact, they claim to be using DSA within their networks and 
have sharing agreements \textit{e.g.} for Amazon's Kindle or Garmin's GPS navigator \cite{CTIA2011}. They oppose to enforced sharing of spectrum. Thus, it is a safer approach for the research community, concerned with increasing the data rate, to further study those approaches with the approval of the incumbents:
\begin{itemize}
\item DSA in cooperative networks. As we formulated in chapter \ref{RSM_chap}, an operator may be willing to access legacy network spectrum without incurring in the cost of upgrading its legacy equipment, or within the context of a previous contractual agreement with other entity such as government equipment or other operators. Depending on the context, protection is likely to lose importance and overall efficiency becomes more important. Therefore, the challenges are significantly different (more of a Pareto efficiency, as in chapter \ref{SPEC_MAN_chap}). %Ser mas concreto. ¿Qué es lo que se debe mejorar?
\item HetNets. As we introduced in chapter \ref{SPEC_MAN_chap}, HetNets also imply an overlay network of small cells over the macro-cell network, with no planned deployment. Due to their unplanned nature and density, it is convenient to equip these small cells with cognitive abilities to avoid interference with macro cell transmissions. As in the previous approach, a substantial difference with classical interweave OSA is that all
the HetNet users are PUs and thus (users of the same operator), and QoS must be guaranteed for both the macro and the small cell users. 
\item Millimeter wave spectrum. This is strongly in line with the industry's desires of more spectrum. The cost and power consumption of its required equipment has decreased and previous propagation issues are now seen as surmountable \cite{Andrews2014}. They key challenge it brings is beamforming (networking through narrow beams). Combined with technologies as MIMO and HetNets, it may become the future \enquote{beachfront spectrum}\footnote{Beachfront spectrum is how the band between 225 Mhz and 3,7 Ghz is called, which is highly valuated because of its propagation characteristics \cite{AdvisorsonScience}.}
\item Advances in MIMO, which further increase the spectral efficiency. Particularly Coordinated multipoint (CoMP) transmission/reception, or massive MIMO.
\item Automated spectrum trading. Summarizing what we explained in the conclusions of chapter \ref{Survey_chap}, the main focus should be on real-time adaptation instead of optimality, applications in practical network environments, alternatives to economic trades such as Cooperative Spectrum Sharing, protection against market failures (collusions, lack of rationality, etc.), and more complex analytic studies (cooperative game theory, stochastic games, learning mechanisms). 
\end{itemize}

Although a deep study in regulation is out of the scope of this thesis, it is impossible to neglect its influence in future research lines. 

\begin{itemize}
\item Policymakers should enforce sharing in some cases. This point should not be seen as contrary to our motivation or the previous points, but more like a complementary and simultaneous approach. There is enough evidence of the benefits of spectrum sharing for the social welfare, and reallocation is not always convenient, as in the federal band. Contrary to the claims of the wireless industry, this may force them to invest in new technologies to improve coexistence. 

\item Regulators could also employ incentive mechanisms to spectrum owners to give up part of this spectrum. This is a controversial point, since most of them got access to spectrum for free. Nevertheless, taking decisions based on historical fairness considerations instead on the current social welfare interests is known as the \enquote{sunk-cost fallacy}.
Enforce other type of obligations to the spectrum owners, such as build-out milestones, roll-out obligations.
\item More research on SU spectrum sharing, as means to ease entrance barriers to the market.
\item Standardization, general interference guidelines, to help remove uncertainty about how much spectrum do each entity gets to use.
\item An intense effort in monitoring activities, but carefully balancing it with speed in the process. Even advocates of the free-market should take into account that transitions from the current situation may need regulator enforcements to avoid harm to social welfare \cite{Yoon2012}. 
\item More complex micro-economic market studies and long-term impact of DSA, competition issues as pointed out by the Report on Radio Spectrum Competition issues, by the European Radio Spectrum Policy Group \cite{ref:RSPG2009}.
\item Especially in Europe, where each country has pursued its own spectrum policies, the regulators have to analyze economic research work in DSA, adopt one of the proposed (and well proven) paradigms and design and enforce suitable policies (with more or less intervention from them).
\end{itemize} 

Also quoting from the same report, \enquote{to date there is limited practical experience} \cite{ref:RSPG2009}. There is a need of studying the practical issues of implementation, test benches, experimental runs, etc. 

Finally, it is also worth mentioning that pursuing an increase in data capacity should be carefully balanced with other objectives such as increasing security, low-latency, and reliability, which would unlock advanced uses of wireless communications. \emph{Security} is a critical factor in next generation networks, given the billions of devices that are expected to join what is called \enquote{the Internet of Things}, some of them affecting the life of millions of people (\textit{e.g.} driverless cars, power grids, medical equipment), coupled with a daily dose of news alerting of important security breaches. DSA also brings its own security threats such as Primary User Emulation attacks \cite{Chen2008}. \emph{Low-latency} applications are grouped around the umbrella term \enquote{the tactile Internet} \cite{Fettweis2014} and range from recreational uses (as in online gaming), to society-transforming ones (e.g. exoskeletons for disabled people). Lastly, also connected with the Internet of Things, and because of increasing levels of complexity in the network architecture, self-healing capabilities \cite{Zhang2013_self} will become key to enhance the reliability of communications as they scale up. This is also relevant in the topic of our thesis, since the wireless industry claims DSA has low scalability and convergence. 


 




