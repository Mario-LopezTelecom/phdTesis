\graphicspath{ {img/SPEC_MAN/} }
\chapter{Introduction}
\section{Motivation}
Wireless communications need, among other resources, radio-electric spectrum, which is finite. 
Government agencies, such as the FCC in the United States or the Electronic Communications Committee (ECC) through national administrations in Europe, grant access to it through fixed, long-term licenses on large geographical areas, in what is called the "command-and-control" management scheme. 
Licenses were given in national lotteries and comparative hearings first, and later through spectrum auctions.
Nevertheless, the ever-growing demand of spectrum due to the popularization of mobile services has left these traditional policies obsolete since almost all the spectrum has already been assigned. %¿Cual es la situacion actual en cuanto a regulacion y por que? Chequear:
% - "Enabling efficient wireless communications..."(2010,paper)
% - "The Spectrum Handbook" (2013, book)
% - "Cognitive Radio Policy and Regulation: Techno-Economic Studies to Facilitate Dynamic Spectrum Access (Signals and Communication Technology" (2014, book)
% - "Digital Crossroads: Telecommunications Law and Policy in the Internet Age" (2013, book)
% - Web del Horizonte2020
However, reports like the 2002 one of the FCC Spectrum Policy Task Force \cite{ref:FCC2002} pointed out that most of the licensed spectrum show very little use, due to the licensee not needing its full capacity and it would be necessary to switch to a more flexible, market-oriented management \cite{ref:Valletti2001} to improve efficiency. % Add figure (the typical with the spectrum assignments).
% Add new documents: 
% - Web del Horizonte2020.
% - "Spectrum Trading in the EU and the US - Shifting Ends and Means".

A dramatic improvement on spectrum usage efficiency could be possible thanks to the development of “cognitive radios” [2], defined as radios capable of gathering infor- mation about their surrounding spectrum environment and adapting their transmission parameters accordingly. In the most typical scenario, unlicensed users or secondary users (SUs) equipped with cognitive radios would look for unused fragments of spectrum anywhere in time, space, frequency and/or power known as spectrum holes, and would use them for their transmissions subject to interference constraints for the protection of licensed or primary users (PUs). This situa- tion corresponds to the “Hierarchical Access Model”, one of the possible coexistence approaches under “Dynamic Spec- trum Access” (DSA) but there are others [4] such as the “Exclusive Use Model”, which considers leasing or selling licenses; and the “Spectrum Commons Model” which as- sumes open sharing with no categories of users.
DSA has recently gained popularity because of the TV digitalization, in order to take advantage of the spectrum opportunities it will generate. Therefore, the last few years have seen some DSA standards such as IEEE 802.22 [9], ECMA-392 [10], 802.11af [12], 802.19.1 [11] and IEEE SCC41 [13].
No matter which model is considered, spectrum shar- ing can annoy licensed operators at present, because it may imply additional costs to them such as changes in their in- frastructures, QoS impairments due to interferences or profit reduction due to increased competition, while they have al- ready paid for their licenses. Among all mechanisms pro- posed to improve spectrum efficiency such as forced resource allocation algorithms, automated spectrum trading has the advantage of providing economic (and/or other) incentives to the entities involved, especially important to licensed op- erators, encouraging their cooperation and promoting moti- vation for future investments in primary spectrum acquisi- tions.

%SPEC- TRADING
Previous works in economic theory are not easily translated to spectrum trading because of the specific features of the traded good. A spectrum market poses the following addi- tional challenges:
– Rapid variations with time
– Imperfect information
– Complex resource allocation considering re-utilization
and heterogeneity of the good
Both spectrum supply and demand changes are related to data traffic intensity which, in general, experiences rapid variations with time. In addition, spectrum characteristics are also highly variable, in particular, availability and chan- nel quality parameters. This implies that the agents making the decisions in the spectrum markets have to be algorith- mic. Automatic transactions are present in financial trading ([14]), but that market does not show the specific issues of spectrum trading.
Imperfect information not only refers to the competitive behavior of agents in a traditional market, by which they are not willing to reveal their true valuation of a good. It also refers to the fact that agents may not have complete or reliable information about the market due to the communi- cation complexity that it would involve, specially in ad-hoc
networks. It entails unfeasible requirements such as, for ex- ample, all entities knowing all channel gains between each other.
Resource allocation becomes more complicated with spec- trum as a traded good because, depending on the mutual in- terference ranges of the entities, it can be spatially reused. It may also be considered that spectrum is an heterogeneous good, in the sense that the same spectrum portion can be valuated differently by each user, depending on its position, technology (spectrum efficiency use), etc.
Rapid variations with time and resource allocation in complex models are tightly interrelated. Optimal resource allocation in complex models requires a computational time that makes it hard to keep up with the rapid changes, which, in turn, may render a solution inefficient because of a time dis-adaptation. Dealing with imperfect information makes optimal resource allocation more complex, even intractable. On the contrary, local estimations could be considered, sac- rificing model optimality for the sake of speed.
\section{Contribution}
\footnote{I will be using the first person plural voice by default throughout my thesis. Most of the ideas and work is a team effort involving my supervisor and other authors. This should not be detrimental of showing off my autonomy, but even my personal decisions where strongly inspired by other' opinions. Research is not done (or should not be done) by a single person in a vaccum. Thus, I consider artificial the use of "I" here. In other parts of this work, such first person plural will also include the reader, primarily when describing mathematical derivations, as a natural way of walking him through the process.}
\section{Methodology}
\section{Structure of the thesis}

