\begin{abstract}
Public regulatory agencies have traditionally granted access to radio-electric spectrum through fixed, long term licenses for large geographical areas. 
The increasing demand of wireless communication has led to an almost fully assigned spectrum. 
However, it is sparsely and unevenly used. 
A dramatic improvement on spectrum usage efficiency is possible through the operation of cognitive radios, capable of gathering information about their surrounding spectrum environment and adapting their transmission parameters accordingly. 
Cognitive radios allow unlicensed users to dynamically exploit unused fragments of spectrum.
This thesis emphasizes the use of mathematical tools such as stochastic modeling, dynamic programming, reinforcement learning, among others, in the development of algorithms for dynamic spectrum access. 
Our main concern in this work is protecting the licensed users from interferences caused by unlicensed users' transmissions, and creating incentives for the licensed users to implement mechanisms for unlicensed access to spectrum. 
Another important guideline of our work is the focus on low-complexity algorithms that can operate at the small time-scale of the spectrum opportunities. %Este claim es dificil de soportar, porque no tenemos estudios serios de computational overhead de ningun algoritmo.
The first part of this thesis focus on protection, through the development of several mechanisms for Opportunistic Spectrum Access, in which unlicensed users scan the spectrum to detect transmission holes so that they can transmit causing as less possible interference to licensed users. In the second part, we focus on the incentives, by developing two frameworks for spectrum trading, in which the licensed users can lease unused bandwidth to unlicensed ones, in exchange of money and relay services, respectively. We also include an extensive survey on the topic, unique in its scope. 
\end{abstract}