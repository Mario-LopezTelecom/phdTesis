\begin{thebibliography}{1}
% Introduction
\bibitem{ref:FCC2002}
{Federal Communications Commission Spectrum Policy Task Force}, ``Report of the
  {S}pectrum {E}fficiency {W}orking {G}roup,'' Tech. Rep., 2002, {ET} {D}ocket
  {N}o. 02-135.

\bibitem{ref:Commission2007}
{Commission of the European Communities}, ``Impact assessment accompanying
  document to {COM}(2007)697, {COM}(2007)698, and {COM}(2007)699,'' Tech. Rep.,
  2007.

\bibitem{ref:Valletti2001}
T.~M. Valletti, ``Spectrum trading,'' \emph{Telecommunications Policy},
  vol.~25, no. 10–11, pp. 655--670, 2001.

\bibitem{ref:Mayo2010}
J.~W. Mayo and S.~Wallsten, ``Enabling efficient wireless communications: The
  role of secondary spectrum markets,'' \emph{Information Economics and
  Policy}, vol.~22, no.~1, pp. 61--72, 2010, {W}ireless {T}echnologies.

\bibitem{ref:ECC2011}
{Electronic Communications Committee (ECC)}, ``Description of practices
  relative to trading of spectrum rights of use,'' Tech. Rep., 2011, {ECC}
  {R}eport 169.

\bibitem{ref:Mitola1999}
J.~Mitola and J.~Maguire, G.Q., ``Cognitive radio: making software radios more
  personal,'' \emph{{IEEE} Personal Commun. Mag.}, vol.~6, no.~4, pp. 13--18,
  Aug. 1999.

\bibitem{ref:Akyildiz2006}
I.~F. Akyildiz, W.-Y. Lee, M.~C. Vuran, and S.~Mohanty, ``{NeXt}
  generation/dynamic spectrum access/cognitive radio wireless networks: A
  survey,'' \emph{Computer Networks}, vol.~50, no.~13, pp. 2127--2159, 2006.

\bibitem{ref:Zhao2007_sur}
Q.~Zhao and B.~Sadler, ``A survey of dynamic spectrum access,'' \emph{{IEEE}
  Signal Process. Mag.}, vol.~24, no.~3, pp. 79--89, May 2007.

\bibitem{ref:Kelly2012} R. B. Kelly and A. Lafrance, \enquote{Spectrum Trading in the EU and the US - Shifting Ends and Means,} in The International Comparative Legal Guide to: Telecommunication Laws and Regulations, Global Legal Group, 2012.

\bibitem{ref:Peha2005} J. Peha, “Approaches to spectrum sharing,” IEEE Commun. Mag., vol. 43, no. 2, pp. 10–12, 2005.

\bibitem{web:Silver2012}
J. Silver-Greenberg, B. Protess, \enquote{Trying to Be Nimble, Knight Capital Stumbles,} DealBook, New York Times. Aug., 2012. [Online]. Available: \url{http://goo.gl/6uPtpp}. Accessed on: Jan., 22, 2015.
%@online{WinNT,
%  author = {MultiMedia LLC},
%  title = {{MS Windows NT} Kernel Description},
%  year = 1999,
%  url = {http://web.archive.org/web/20080207010024/http://www.808multimedia.com/winnt/kernel.htm},
%  urldate = {2010-09-30}
%}

\bibitem{ref:Bkassiny2013} M. Bkassiny, Y. Li, and S. K. Jayaweera, \enquote{A Survey on Machine-Learning Techniques in Cognitive Radios,} IEEE Commun. Surv. Tutorials, vol. 15, no. 3, pp. 1136–1159, Oct. 2013.

\bibitem{ref:White2012} J. M. White, \enquote{Bandit Algorithms for Website Optimization.} Sebastopol, CA: O’Reilly, 2012.

\bibitem{ref:MATLAB} The MathWorks, Inc., MATLAB Release 2014b [Computer software]. Natick, Massachusetts, United States, 2014. Available: \url{http://www.mathworks.com/products/matlab/}.

\bibitem{ref:Mathematica} Wolfram Research, Inc., Mathematica, Version 10.0 [Computer software]. Champaign, IL, 2014. Available: \url{http://www.wolfram.com/mathematica/}.

\bibitem{ref:RStudio} RStudio, RStudio: Integrated development environment for R (Version 0.98.1091) [Computer software]. Boston, MA, 2014.
Available: \url{http://www.rstudio.org/}.

\bibitem{ref:SciPy} E. Jones, T. Oliphant, P. Peterson, et al., SciPy: Open Source Scientific Tools for Python, Version 0.15.0 [Computer software], 2015. Available: \url{http://www.scipy.org}.

\bibitem{ref:Omnet} Andr\'{a}s Varga, OMNeT++, Version 4.6 [Computer Software], 2014. Available: \url{http://www.omnetpp.org}.

\bibitem{ref:Alcaraz2015_BD}
J.~Alcaraz, M.~L\'{o}pez-Mart\'{i}nez, J.~Vales-Alonso, and J.~Garcia-Haro,
  ``Background detection of primary user activity in opportunistic spectrum
  access,'' \emph{{IEEE} International Conference on Communications {ICC}},
  2015, to be published.
  
\bibitem{ref:Alcaraz2014_BR}
J.~Alcaraz, M.~L\'{o}pez-Mart\'{i}nez, J.~Vales-Alonso, and J.~Garcia-Haro, ``Bandwidth reservation as a coexistence strategy in opportunistic
  spectrum access environments,'' \emph{{IEEE} J. Sel. Areas Commun.}, vol.~32,
  no.~3, pp. 478--488, Mar. 2014.

\bibitem{ref:Alcaraz2015_SPEC_MAN}
J.~Alcaraz, M.~L\'{o}pez-Mart\'{i}nez, J.~Vales-Alonso, and J.~Garcia-Haro, ``Uncertainty-Aware Opportunistic Spectrum Access in Coexistence-Friendly Systems,'' \emph{{IEEE} Transactions on Cognitive
  Communications and Networking}, 2015, to be submitted for publication.

\bibitem{ref:Alcaraz2014_RSM}
J.~Alcaraz, J.~Ayala-Romero, M.~L\'{o}pez-Mart\'{i}nez, and J.~Vales-Alonso,
  ``Combining dual tessellation and temporal opportunities for spectrum reuse
  in cellular systems,'' in \emph{11th International Symposium on Wireless
  Communications Systems ({ISWCS})}, Aug. 2014, pp. 486--490.

\bibitem{ref:Alcaraz2015_RSM}
J.~Alcaraz, J.~Ayala-Romero, M.~L\'{o}pez-Mart\'{i}nezz, and J.~Vales-Alonso, ``Response surface methodology for efficient spectrum reuse in cellular
  networks,'' in \emph{{IEEE} International Conference on Communications
  {ICC}}, 2015, to be published.

\bibitem{ref:Mario2014}
M.~L\'{o}pez-Mart\'{i}nez, J.~J. Alcaraz, J.~Vales-Alonso, and J.~Garcia-Haro,
  ``Automated spectrum trading mechanisms: understanding the big picture,''
  \emph{Wireless Networks}, 2014.

\bibitem{ref:Alcaraz2012}
J.~Alcaraz, M.~L\'{o}pez-Mart\'{i}nez, J.~Vales-Alonso, and J.~Garcia-Haro, ``An {MDP}
  framework for centralized dynamic spectrum auction,'' in \emph{35th {IEEE}
  Sarnoff Symposium ({SARNOFF})}, May 2012, pp. 1--5.

\bibitem{ref:Mario2015_MAB_CSSA_1}
J.~Alcaraz, J.~Ayala-Romero, M.~L\'{o}pez-Mart\'{i}nez, and J.~Vales-Alonso,
  ``Multi-armed bandits with dependent arms for cooperative spectrum sharing,''
  in \emph{{IEEE} International Conference on Communications {ICC}}, 2015, to be published.

\bibitem{ref:Mario2015_MAB_CSSA_2}
M.~L\'{o}pez-Mart\'{i}nez, J.~J. Alcaraz, B.~L., and M.~Zorzi, ``A superprocess with
  upper confidence bounds for cooperative spectrum sharing,'' \emph{{IEEE}
  Trans. Mobile Comput.}, 2015, submitted for publication.
% BD

\bibitem{ref:Mihaly2013} M. Csikszentmihalyi \emph{Creativity: The Psychology of Discovery and Invention} \hskip 1em plus 0.5em minus 0.4em\relax New York, NY: Harper Perennial, Reprint edition, 2013.

\bibitem{ref:SurveySensing} M. Masonta, M. Mzyece, N. Ntlatlapa, ``Spectrum Decision in Cognitive Radio Networks: A Survey,'' \emph{Communications Surveys \& Tutorials, IEEE }, vol.PP, no.99, pp.1-20, 0 doi: 10.1109/SURV.2012.111412.00160.
\bibitem{ref:TransSignalProc} Q. C. Zhao, S. Geirhofer, L. Tong, and B. M. Sadler, ``Opportunistic spectrum access via periodic channel sensing,'' \emph{IEEE Trans. Signal Process.}, vol. 56, no. 2, pp. 785-796, Feb. 2008.
\bibitem{ref:JSAC2007} Q. Zhao, L. Tong, A. Swami, and Y. Chen, ``Decentralized cognitive MAC for opportunistic spectrum access in ad hoc networks: a POMDP framework,'' \emph{IEEE J. Sel. Areas Commun.}, vol. 25, no. 3, pp. 589-600, Apr. 2007.
\bibitem{ref:TransWire2008} Y. C. Liang, Y. H. Zeng, E. Peh, and A. T. Hoang, ``Sensing-throughput tradeoff for cognitive radio networks,'' \emph{IEEE Trans. Wireless Commun.}, vol. 7, no. 4, pp. 1326-1336, Apr. 2008.
\bibitem{ref:TransWire2009} X. W. Zhou, J. Ma, G. Y. Li, Y. H. Kwon, and A. C. K. Soong, ``Probability-based optimization of inter-sensing duration and power control in cognitive radio,'' \emph{IEEE Trans. Wireless Commun.}, vol. 8, no.
10, pp. 4922-4927, Apr. 2009.
%\bibitem{ref:Conference2007} A. Ghasemi and E. S. Sousa, ``Optimization of spectrum sensing for opportunistic spectrum access in cognitive radio networks,'' \emph{in Proc. 2007 IEEE Consumer Commun. Netw. Conf.}, pp. 1022–1026.
\bibitem{ref:TransWireOct2008} W. Y. Lee and I. F. Akyildiz, ``Optimal spectrum sensing framework for cognitive radio network,'' IEEE Trans. Wireless Commun., vol. 7, no. 10, pp. 3845-3857, Oct. 2008.
\bibitem{ref:TransWire2012} J. Zhang, L. Qi, H. Zhu, ``Optimization of MAC Frame Structure for Opportunistic Spectrum Access,'' \emph{Wireless Communications, IEEE Transactions on }, vol.11, no.6, pp.2036-2045, June 2012.
\bibitem{ref:TransNet2012} E. Jung, X. Liu, ``Opportunistic Spectrum Access in Multiple-Primary-User Environments Under the Packet Collision Constraint,'' \emph{Networking, IEEE/ACM Transactions on }, vol.20, no.2, pp.501-514, April 2012.

%\bibitem{ref:comparativaMAC} P. Pawelczak, \textit{et. al.}, ``Performance Analysis of Multichannel Medium Access Control Algorithms for Opportunistic Spectrum Access,'' \emph{Vehicular Technology, IEEE Transactions on }, vol.58, no.6, pp.3014-3031, July 2009.
%\bibitem{ref:SurveyMACs} A. De Domenico, E. Calvanese Strinati and M.G. Di Benedetto, ``A Survey on MAC Strategies for Cognitive Radio Networks,'' \emph{Communications Surveys \& Tutorials, IEEE }, vol. 14, no. 1, pp. 21-44, 2012.
%\bibitem{ref:HCMAC} J. Jia, Q. Zhang, X. Shen, ``HC-MAC: A Hardware-Constrained Cognitive MAC for Efficient Spectrum Management,'' \emph{Selected Areas in Communications, IEEE Journal on }, vol.26, no.1, pp.106-117, Jan. 2008.
%\bibitem{ref:Gao2011} L. Gao, X. Wang, Y. Xu and Q. Zhang, ``Spectrum trading in cognitive radio networks: a contract-theoretic modeling approach,'' \emph{IEEE J. Selected Areas in Comm.}, 2011.
%\bibitem{ref:Yu2010} H. Yu, L. Gao, Z. Li. X. Wang, and E. Hossain, ``Pricing for uplink power control in cognitive radio networks,''\emph{ IEEE Trans. Veh. Technol.}, 2010.
%\bibitem{ref:Yang2011} L. Yang, H. Kim, J. Zhang, M. Chiang, and C. W. Tan, ``Pricing-based spectrum access control in cognitive radio networks with random access,'' in \textit{IEEE INFOCOM}, 2011.
%\bibitem{ref:CircuitSwitched} J.P. Vasseur, M. Pickavet, and P. Demeester. ``Network recovery: Protection and Restoration of Optical, SONET-SDH, IP, and MPLS.'' Morgan Kaufmann, 2004.
%\bibitem{ReservationHandoff} R. Ramjee, R. Nagarajan, and D. Towsley. ``On optimal call admission
%control in cellular networks,'' \emph{IEEE INFOCOM}, 1996.
%\bibitem{ReservationHandoff2} J. Vazquez-Avila, F.A. Cruz-Perez and L. Ortigoza-Guerrero, ``Performance analysis of fractional guard channel policies in mobile
%cellular networks,'' \emph{IEEE Transactions on Wireless Communications}, vol. 5, no.2, Feb. 2006.
%\bibitem{ref:Surveycognitive1} Y.-C., Liang, K.-C. Chen, G.Y. Li, and P. Mähönen, ``Cognitive radio networking and communications: An overview,''\emph{ IEEE Trans. Veh. Technol.} vol. 60, no. 7, pp. 3386-3407, Sept. 2011.
%\bibitem{ref:Surveycognitive2} B. Wang, and K.J. Ray Liu, ``Advances in cognitive radio networks: A survey,'' \emph{IEEE Journal of Selected Topics in Signal Processing}, vol. 5n no. 1 pp. 5-23, Feb. 2011.
%\bibitem{ref:Reservation} X. Zhu, L. Shen, and T.-S.P. Yum., ``Analysis of cognitive radio spectrum access with optimal channel reservation,'' \emph{IEEE  Comm. Letters}, vol. 11, no. 4, 304-306, Apr. 2011.
%\bibitem{ref:Reservation2} J. Martinez-Bauset, V. Pla, D. Pacheco-Paramo, ``Comments on ``analysis of cognitive radio spectrum access with optimal channel reservation'','' \emph{IEEE  Comm. Letters}, vol.13, no.10, pp.739, Oct. 2009.
%\bibitem{ref:Reservation3} W. Ahmed, J. Gao, H. Suraweera, M. Faulkner, ``Comments on ``analysis of cognitive radio spectrum access with optimal channel reservation'','' \emph{IEEE Transactions on Wireless Communications}, vol.8, no.9, pp. 4488-4491, Sept. 2009.
%\bibitem{ref:Reservation3b} Lai, Jin, Ren Ping Liu, Eryk Dutkiewicz, and Rein Vesilo, ``Optimal Channel Reservation in Cooperative Cognitive Radio Networks,'' in \emph{IEEE 73rd Vehicular Technology Conference (VTC Spring)}, pp. 1-6, 2011.
%\bibitem{ref:Reservation4} P.K. Tang, Y.H. Chew, L.C. Ong, M.K.Haldar, ``Performance of secondary radios in spectrum sharing with prioritized primary access,'' in \emph{Military Communications Conference, 2006}, IEEE MILCOM 2006, pp. 1-7. 2006.
%\bibitem{ref:Reservation5} G. Wu, P.Ren, and Q. Du, ``Recall-Based Dynamic Spectrum Auction with the Protection of Primary Users,'' \emph{IEEE Journal on Selected Areas in Communications}, vol. 30, no. 10 pp. 2070-2081, 2012.
%\bibitem{ref:Coexistence} Chen Sun; G.P. Villardi, Zhou Lan, Y.D. Alemseged, H.N. Tran, H. Harada,  ``Optimizing the Coexistence Performance of Secondary-User Networks Under Primary-User Constraints for Dynamic Spectrum Access,''\emph{ IEEE Trans. Veh. Technol.}, vol.61, no.8, pp.3665-3676, Oct. 2012.
%%\bibitem{ref:QoS} P. Paweczak, \textit{et. al}, ``Quality of service assessment of opportunistic spectrum access: A medium access control approach,'' \emph{IEEE Wireless Commun.}, vol. 15, no. 5, pp. 20-29, Oct. 2008.
%\bibitem{ref:DiscoveringOpportunities} H. Kim. and K.G. Shin, ``Efficient Discovery of Spectrum Opportunities with MAC-Layer Sensing in Cognitive Radio Networks,'' \emph{Mobile Computing, IEEE Transactions on }, vol.7, no.5, pp.533-545, May 2008.

\bibitem{ref:StoppingRule} H. T. Cheng, H. Shan and W. Zhuang; , ``Stopping Rule-Driven Channel Access in Multi-Channel Cognitive Radio Networks,'' \emph{2011 IEEE International Conference on Communications (ICC)}, pp.1-6, 5-9 June 2011.
%\bibitem{ref:PacketConstraints} E. Jung and X. Liu, ``Opportunistic Spectrum Access in Multiple-Primary-User Environments Under the Packet Collision Constraint,'' \emph{Networking, IEEE/ACM Transactions on }, \textit{in press} doi: 10.1109/TNET.2011.2164933.
%\bibitem{ref:POMDP} Q. Zhao, L. Tong, A. Swami and Y. Chen, ``Decentralized cognitive MAC for opportunistic spectrum access in ad hoc networks: A POMDP framework,'' \emph{Selected Areas in Communications, IEEE Journal on }, vol.25, no.3, pp.589-600, April 2007.
\bibitem{ref:OptimalStrategies} S. Huang, X. Liu and Z. Ding, ``Optimal Transmission Strategies for Dynamic Spectrum Access in Cognitive Radio Networks,'' \emph{Mobile Computing, IEEE Transactions on }, vol.8, no.12, pp.1636-1648, Dec. 2009.
\bibitem{ref:OpportunisticAccess} S. Huang, X. Liu and Z. Ding, ``Opportunistic Spectrum Access in Cognitive Radio Networks,'' \emph{in Proc. IEEE INFOCOM 2008}, pp.1427-1435, 13-18 April 2008.
\bibitem{ref:MultichannelMultistage} W. Gabran, P. Pawelczak and D. Cabric, ``Throughput and Collision Analysis of Multichannel Multistage Spectrum Sensing Algorithms,'' \emph{Vehicular Technology, IEEE Transactions on }, vol.60, no.7, pp.3309-3323, Sept. 2011.
\bibitem{ref:SensingMultichannel} J. Park, P. Pawelczak, and D. Cabric, ``Performance of Joint Spectrum Sensing and MAC Algorithms for Multichannel Opportunistic Spectrum Access Ad Hoc Networks,'' \emph{ IEEE Transactions on Mobile Computing}, \textit{in press} doi: 10.1109/TMC.2010.186
\bibitem{ref:JSAC2013_PUreturn}
\bibitem{ref:JSAC2013_PUtraffic}













%BR
\bibitem{ref:comparativaMAC} P. Pawelczak, \textit{et. al.}, ``Performance Analysis of Multichannel Medium Access Control Algorithms for Opportunistic Spectrum Access,'' \emph{Vehicular Technology, IEEE Transactions on }, vol.58, no.6, pp.3014-3031, July 2009.
\bibitem{ref:SurveyMACs} A. De Domenico, E. Calvanese Strinati and M.G. Di Benedetto, ``A Survey on MAC Strategies for Cognitive Radio Networks,'' \emph{Communications Surveys \& Tutorials, IEEE }, vol. 14, no. 1, pp. 21-44, 2012.
\bibitem{ref:HCMAC} J. Jia, Q. Zhang, X. Shen, ``HC-MAC: A Hardware-Constrained Cognitive MAC for Efficient Spectrum Management,'' \emph{Selected Areas in Communications, IEEE Journal on }, vol.26, no.1, pp.106-117, Jan. 2008.

\bibitem{ref:CircuitSwitched} J.P. Vasseur, M. Pickavet, and P. Demeester. ``Network recovery: Protection and Restoration of Optical, SONET-SDH, IP, and MPLS.'' Morgan Kaufmann, 2004.
\bibitem{ReservationHandoff} R. Ramjee, R. Nagarajan, and D. Towsley. ``On optimal call admission
control in cellular networks,'' \emph{IEEE INFOCOM}, 1996.
\bibitem{ReservationHandoff2} J. Vazquez-Avila, F.A. Cruz-Perez and L. Ortigoza-Guerrero, ``Performance analysis of fractional guard channel policies in mobile
cellular networks,'' \emph{IEEE Transactions on Wireless Communications}, vol. 5, no.2, Feb. 2006.
\bibitem{ref:Surveycognitive1} Y.-C., Liang, K.-C. Chen, G.Y. Li, and P. Mähönen, ``Cognitive radio networking and communications: An overview,''\emph{ IEEE Trans. Veh. Technol.} vol. 60, no. 7, pp. 3386-3407, Sept. 2011.
\bibitem{ref:Surveycognitive2} B. Wang, and K.J. Ray Liu, ``Advances in cognitive radio networks: A survey,'' \emph{IEEE Journal of Selected Topics in Signal Processing}, vol. 5n no. 1 pp. 5-23, Feb. 2011.
\bibitem{ref:Reservation} X. Zhu, L. Shen, and T.-S.P. Yum., ``Analysis of cognitive radio spectrum access with optimal channel reservation,'' \emph{IEEE  Comm. Letters}, vol. 11, no. 4, 304-306, Apr. 2011.
\bibitem{ref:Reservation2} J. Martinez-Bauset, V. Pla, D. Pacheco-Paramo, ``Comments on ``analysis of cognitive radio spectrum access with optimal channel reservation'','' \emph{IEEE  Comm. Letters}, vol.13, no.10, pp.739, Oct. 2009.
\bibitem{ref:Reservation3} W. Ahmed, J. Gao, H. Suraweera, M. Faulkner, ``Comments on ``analysis of cognitive radio spectrum access with optimal channel reservation'','' \emph{IEEE Transactions on Wireless Communications}, vol.8, no.9, pp. 4488-4491, Sept. 2009.
\bibitem{ref:Reservation3b} Lai, Jin, Ren Ping Liu, Eryk Dutkiewicz, and Rein Vesilo, ``Optimal Channel Reservation in Cooperative Cognitive Radio Networks,'' in \emph{IEEE 73rd Vehicular Technology Conference (VTC Spring)}, pp. 1-6, 2011.
\bibitem{ref:Reservation4} P.K. Tang, Y.H. Chew, L.C. Ong, M.K.Haldar, ``Performance of secondary radios in spectrum sharing with prioritized primary access,'' in \emph{Military Communications Conference, 2006}, IEEE MILCOM 2006, pp. 1-7. 2006.
\bibitem{ref:Reservation5} G. Wu, P.Ren, and Q. Du, ``Recall-Based Dynamic Spectrum Auction with the Protection of Primary Users,'' \emph{IEEE Journal on Selected Areas in Communications}, vol. 30, no. 10 pp. 2070-2081, 2012.
\bibitem{ref:BookCognitive} E. Biglieri, et. al., ``Principles of Cognitive Radio,'' Cambridge University Press, 2012.
\bibitem{ref:Coexistence} Chen Sun; G.P. Villardi, Zhou Lan, Y.D. Alemseged, H.N. Tran, H. Harada,  ``Optimizing the Coexistence Performance of Secondary-User Networks Under Primary-User Constraints for Dynamic Spectrum Access,''\emph{ IEEE Trans. Veh. Technol.}, vol.61, no.8, pp.3665-3676, Oct. 2012.
\bibitem{ref:DiscoveringOpportunities} H. Kim. and K.G. Shin, ``Efficient Discovery of Spectrum Opportunities with MAC-Layer Sensing in Cognitive Radio Networks,'' \emph{Mobile Computing, IEEE Transactions on }, vol.7, no.5, pp.533-545, May 2008.
\bibitem{ref:PacketConstraints} E. Jung and X. Liu, ``Opportunistic Spectrum Access in Multiple-Primary-User Environments Under the Packet Collision Constraint,'' \emph{Networking, IEEE/ACM Transactions on }, \textit{in press} doi: 10.1109/TNET.2011.2164933.
\bibitem{ref:POMDP} Q. Zhao, L. Tong, A. Swami and Y. Chen, ``Decentralized cognitive MAC for opportunistic spectrum access in ad hoc networks: A POMDP framework,'' \emph{Selected Areas in Communications, IEEE Journal on }, vol.25, no.3, pp.589-600, April 2007.

\bibitem{ref:lowComplexity} Li, Yang, et al. ``Optimal Myopic Sensing and Dynamic Spectrum Access in Cognitive Radio Networks with Low-Complexity Implementations,'' Wireless Communications, IEEE Transactions on 11.7 (2012): 2412-2423.
\bibitem{ref:MultichannelMAC} W.S. Jeon, J.A. Han, and D. G. Jeong, ``A novel MAC scheme for multichannel cognitive radio ad hoc networks,'' \emph{IEEE Transactions on Mobile Computing}, vol. 11, no. 6, pp. 922-934, June 2012.
\bibitem{ref:MultichannelMAC2} D. Xu, E.J. Dan, and X. Liu, ``Efficient and Fair Bandwidth Allocation in Multichannel Cognitive Radio Networks,'' \emph{IEEE Transactions on Mobile Computing}, vol. 11, no.8, pp. 1372-1385, Aug. 2012.
\bibitem{ref:PSD} W. Rhee and J. M. Cioffi, ``Increase in capacity of multiuser OFDM
system using dynamic subchannel allocation,'' \emph{in Proc. IEEE VTC}, vol. 2, pp. 1085-1089, May 2000.
\bibitem{ref:OFDMtutorial} S. Sadr, A. Anpalagan and K. Raahemifar, ``Radio Resource Allocation Algorithms for the Downlink of Multiuser OFDM Communication Systems,'' \emph{IEEE Communications Surveys \& Tutorials}, Vol. 11, No. 3, 2009.
\bibitem{ref:Gelabert}
X. Gelabert, O. Sallent, J. Pérez-Romero, and R. Agustí, ``Flexible Spectrum Access for Opportunistic Secondary Operation in Cognitive Radio Networks,'' \emph{IEEE Transactions on Communications}, vol. 59, no. 10, pp. 2659-2664, Oct. 2011.
\bibitem{ref:Pla}   
L. Jiao, F.Y. Li, and V. Pla., ``Modeling and Performance Analysis of Channel Assembling in Multichannel Cognitive Radio Networks With Spectrum Adaptation,'' \emph{IEEE Transactions on Vehicular Technology}, vol. 61, no. 6, pp. 2686-2697, Jul. 2012.
\bibitem{ref:Modelling} S. Tang and B. L. Mark, ``Modelling and analysis of opportunistic spectrum sharing with unreliable spectrum sensing,'' \emph{IEEE Trans. Wireless Commun.}, vol. 8, no. 4, pp. 1934-1943, Apr. 2009.
\bibitem{ref:Bertsekas}
D. Bertsekas, J. Tsitsiklis, ``Introduction to Probability, 2nd Edition'' Athenea Scientific, 2008. 
\bibitem{ref:Solomon}
Solomon, H. ``Geometric Probability,'' Philadelphia, PA: SIAM, 1978.













% SPEC_MAN
\bibitem{ref:Alcaraz2013}
J.~J. Alcaraz, M.~Lopez-Martinez, J.~Vales-Alonso, and J.~Garcia-Haro,
  ``Bandwidth reservation as a coexistence strategy in opportunistic spectrum
  access environments,'' \emph{{IEEE} J. Sel. Areas Commun.}, vol.~32, no.~3,
  pp. 478--488, March 2014.

\bibitem{ref:ElSawy2013}
H.~ElSawy and E.~Hossain, ``Channel assignment and opportunistic spectrum
  access in two-tier cellular networks with cognitive small cells,'' in
  \emph{IEEE Global Communications Conference ({GLOBECOM})}, Dec 2013, pp.
  4477--4482.

\bibitem{ref:Gavrilovska2015}
L.~Gavrilovska, D.~Denkovski, V.~Rakovic, and M.~Angjelicinoski, ``Medium
  access control protocols in cognitive radio networks,'' in \emph{Cognitive
  Radio and Networking for Heterogeneous Wireless Networks}, ser. Signals and
  Communication Technology.\hskip 1em plus 0.5em minus 0.4em\relax Springer
  International Publishing, 2015, pp. 109--149.

\bibitem{ref:Domenico2012}
A.~De~Domenico, E.~Strinati, and M.~Di~Benedetto, ``A survey on {MAC}
  strategies for cognitive radio networks,'' \emph{Communications Surveys
  Tutorials, IEEE}, vol.~14, no.~1, pp. 21--44, First Quarter 2012.

\bibitem{ref:Pawelczak2008}
P.~Pawelczak, S.~Pollin, H.-S. So, A.~Bahai, R.~Prasad, and R.~Hekmat,
  ``Quality of service assessment of opportunistic spectrum access: a medium
  access control approach,'' \emph{{IEEE} Wireless Commun. Mag.}, vol.~15,
  no.~5, pp. 20--29, October 2008.

\bibitem{ref:Pawelczak2009}
P.~Pawelczak, S.~Pollin, H.-S. So, A.~Bahai, R.~Venkatesha~Prasad, and
  R.~Hekmat, ``Performance analysis of multichannel medium access control
  algorithms for opportunistic spectrum access,'' \emph{{IEEE} Trans. Veh.
  Technol.}, vol.~58, no.~6, pp. 3014--3031, July 2009.

\bibitem{ref:Sun2012}
C.~Sun, G.~Villardi, Z.~Lan, Y.~Alemseged, H.~Tran, and H.~Harada, ``Optimizing
  the coexistence performance of secondary-user networks under primary-user
  constraints for dynamic spectrum access,'' \emph{{IEEE} Trans. Veh.
  Technol.}, vol.~61, no.~8, pp. 3665--3676, Oct 2012.

\bibitem{ref:Gabran2011}
W.~Gabran, P.~Pawelczak, and D.~Cabric, ``Throughput and collision analysis of
  multichannel multistage spectrum sensing algorithms,'' \emph{{IEEE} Trans.
  Veh. Technol.}, vol.~60, no.~7, pp. 3309--3323, Sept 2011.

\bibitem{ref:Jung2012}
E.~Jung and X.~Liu, ``Opportunistic spectrum access in multiple-primary-user
  environments under the packet collision constraint,'' \emph{{IEEE/ACM} Trans.
  Netw.}, vol.~20, no.~2, pp. 501--514, April 2012.

\bibitem{ref:Zhao2007_dec}
Q.~Zhao, L.~Tong, A.~Swami, and Y.~Chen, ``Decentralized cognitive {MAC} for
  opportunistic spectrum access in ad hoc networks: A {POMDP} framework,''
  \emph{{IEEE} J. Sel. Areas Commun.}, vol.~25, no.~3, pp. 589--600, April
  2007.

\bibitem{ref:Huang2009}
S.~Huang, X.~Liu, and Z.~Ding, ``Optimal transmission strategies for dynamic
  spectrum access in cognitive radio networks,'' \emph{{IEEE} Trans. Mobile
  Comput.}, vol.~8, no.~12, pp. 1636--1648, Dec 2009.

\bibitem{ref:Huang2008_opp}
------, ``Opportunistic spectrum access in cognitive radio networks,'' in
  \emph{The 27th Conference on Computer Communications. {IEEE} {INFOCOM}},
  April 2008.

\bibitem{ref:Mao2010}
X.~Mao, H.~Ji, V.~Leung, and M.~Li, ``Performance enhancement for unlicensed
  users in coordinated cognitive radio networks via channel reservation,'' in
  \emph{IEEE Global Telecommunications Conference {GLOBECOM}}, Dec 2010, pp.
  1--5.

\bibitem{ref:Tang2009_per}
P.~K. Tang, Y.~H. Chew, W.-L. Yeow, and L.~C. Ong, ``Performance comparison of
  three spectrum admission control policies in coordinated dynamic spectrum
  sharing systems,'' \emph{{IEEE} Trans. Veh. Technol.}, vol.~58, no.~7, pp.
  3674--3683, Sept 2009.

\bibitem{ref:Mitola2000}
J.~Mitola, ``Cognitive radio: An integrated agent architecture for software
  defined radio,'' Ph.D. dissertation, Royal Institute of Technology (KTH),
  Kista, Sweden, 2000.

\bibitem{ref:Lima2012}
C.~de~Lima, M.~Bennis, and M.~Latva-aho, ``Coordination mechanisms for
  self-organizing femtocells in two-tier coexistence scenarios,'' \emph{{IEEE}
  Trans. Wireless Commun.}, vol.~11, no.~6, pp. 2212--2223, June 2012.

\bibitem{ref:Wu2012}
G.~Wu, P.~Ren, and Q.~Du, ``Recall-based dynamic spectrum auction with the
  protection of primary users,'' \emph{{IEEE} J. Sel. Areas Commun.}, vol.~30,
  no.~10, pp. 2070--2081, November 2012.

\bibitem{ref:Biglieri2012}
E.~Biglieri, A.~J. Goldsmith, L.~J. Greenstein, N.~B. Mandayam, and H.~V. Poor,
  \emph{Principles of Cognitive Radio}.\hskip 1em plus 0.5em minus 0.4em\relax
  Cambridge University Press, 2012.

\bibitem{ref:Maille2009}
P.~Maille and B.~Tuffin, ``Price war with partial spectrum sharing for
  competitive wireless service providers,'' in \emph{IEEE Global
  Telecommunications Conference, {GLOBECOM}}, Nov 2009, pp. 1--6.

\bibitem{ref:Cheng2011}
H.~T. Cheng, H.~Shan, and W.~Zhuang, ``Stopping rule-driven channel access in
  multi-channel cognitive radio networks,'' in \emph{{IEEE} International
  Conference on Communications ({ICC})}, June 2011, pp. 1--6.

\bibitem{ref:Park2011}
J.~Park, P.~Paweczak, and D.~Cabric, ``Performance of joint spectrum sensing
  and {MAC} algorithms for multichannel opportunistic spectrum access ad hoc
  networks,'' \emph{{IEEE} Trans. Mobile Comput.}, vol.~10, no.~7, pp.
  1011--1027, July 2011.

\bibitem{ref:Kim2008}
H.~Kim and K.~Shin, ``Efficient discovery of spectrum opportunities with
  {MAC}-layer sensing in cognitive radio networks,'' \emph{{IEEE} Trans. Mobile
  Comput.}, vol.~7, no.~5, pp. 533--545, May 2008.

\bibitem{ref:Tang2009_mod}
S.~Tang and B.~Mark, ``Modeling and analysis of opportunistic spectrum sharing
  with unreliable spectrum sensing,'' \emph{{IEEE} Trans. Wireless Commun.},
  vol.~8, no.~4, pp. 1934--1943, April 2009.

\bibitem{ref:Jia2008_HC}
J.~Jia, Q.~Zhang, and X.~Shen, ``{HC-MAC}: A hardware-constrained cognitive mac
  for efficient spectrum management,'' \emph{{IEEE} J. Sel. Areas Commun.},
  vol.~26, no.~1, pp. 106--117, Jan 2008.

\bibitem{ref:Li2012}
Y.~Li, S.~Jayaweera, M.~Bkassiny, and K.~Avery, ``Optimal myopic sensing and
  dynamic spectrum access in cognitive radio networks with low-complexity
  implementations,'' \emph{{IEEE} Trans. Wireless Commun.}, vol.~11, no.~7, pp.
  2412--2423, July 2012.

\bibitem{ref:Bertsekas2012}
D.~Bertsekas, \emph{Dynamic Programming and Optimal Control}, 4th~ed.\hskip 1em
  plus 0.5em minus 0.4em\relax Athena Scientific, 2012, vol.~I.

\bibitem{ref:Bolch2006}
D.~Bolch, \emph{Queueing Networks and Markov Chains : Modeling and Performance
  Evaluation With Computer Science Applications}, 2nd~ed.\hskip 1em plus 0.5em
  minus 0.4em\relax Wiley-Interscience, 2006.

\bibitem{ref:Puterman2005}
M.~L. Puterman, \emph{Markov Decision Processes: Discrete Stochastic Dynamic
  Programming}.\hskip 1em plus 0.5em minus 0.4em\relax Wiley-Interscience,
  2005.










%RSM
\bibitem{ref:Goldsmith2} E. Biglieri et.al., \emph{Principles of Cognitive Radio}, Cambridge University Press, 2013.

\bibitem{ref:Hybrid} D. Tuan, B.L. Mark, ``Joint spatial-temporal spectrum sensing for cognitive radio networks,'' , \emph{IEEE Transactions on Vehicular Technology}, vol.59, no.7, pp.3480,3490, Sept. 2010

\bibitem{ref:Hybrid2} Q. Wu et.al., ``Spatial-Temporal Opportunity Detection for Spectrum-Heterogeneous Cognitive Radio Networks: Two-Dimensional Sensing,'' \emph{IEEE Trans. on Wireless Commun.}, vol.12, no.2, pp.516-526, Feb. 2013.

\bibitem{ref:HybridVehicular} D. Guoru, et.al., ``Joint exploration and exploitation of spatial-temporal spectrum hole for cognitive vehicle radios,'' \emph{2011 IEEE International Conference on Signal Processing, Communications and Computing (ICSPCC)}, pp.1-4, Sep. 2011.

\bibitem{ref:Hybrid3}  M. G. Khoshkholgh, K. Navaie, and H. Yanikomeroglu, ``Access strategies for spectrum sharing in fading environment: overlay, underlay and mixed,'' \emph{IEEE Trans. Mobile Comput.}, vol. 9, no. 12, pp. 1780-1793, Dec. 2010.

\bibitem{ref:Cellular} E. G. Larsson, M. Skoglund, ``Cognitive radio in a frequency-planned environment: some basic limits,'' \emph{IEEE Transactions on Wireless Communications}, vol.7, no.12, pp.4800-06, Dec. 2008.

\bibitem{ref:Cellular2} E. Axell, E.G. Larsson, D. Danev, ``Capacity considerations for uncoordinated communication in geographical spectrum holes,'' \emph{Physical Communication}, vol. 2, no.1, pp- 3-9, Mar. 2009.

\bibitem{ref:DualTesselation} J. J. Alcaraz, J. A. Ayala-Romero, M. Lopez-Martinez, J. Vales-Alonso, ``Combining Dual Tesselation and Temporal Access for Spectrum Reuse in Cellular Systems'', \emph{11th International Symposium on Wireless Communication Systems}, Aug. 2014.

\bibitem{ref:Goldsmith} A. Goldsmith, \emph{Wireless Communications}, Cambridge University Press, 2005.

\bibitem{ref:RSM_book} K. Marti, \emph{Stochastic Optimization Methods}, Springer-Verlag, Berlin-Heidelberg, 2008.

\bibitem{ref:RSM_restr} E. Angün et.al., \emph{Response surface methodology with stochastic constraints for expensive simulation}, Journal of the Operational Research Society, 60 (6) (2009), pp. 735-746.

% Survey
\bibitem{ref:BWang2011} Wang, B., \& Liu, K. (2011). Advances in cognitive radio networks: A survey. \textit{IEEE Selected Topics in Signal Processing}, 5(1), 5–23. doi:10.1109/JSTSP.2010.2093210

\bibitem{ref:Yoon2012} Yoon, H., Hwang, J., \& Weiss, M. B. H. (2012). An analytic research on secondary-spectrum trading mechanisms based on technical and market changes. \textit{Computer Networks}, 56(1), 3–19. doi:10.1016/j.comnet.2011.05.017
\bibitem{ref:YZhao2009} Zhao, Y., Mao, S., Neel, J., \& Reed, J. (2009). Performance evaluation of cognitive radios: Metrics, utility functions, and methodology. \textit{Proceedings of the IEEE}, 97(4).
\bibitem{ref:802} Niyato, D., Hossain, E., \& Han, Z. (2009). Dynamic spectrum access in IEEE 802.22-based cognitive wireless networks: a game theoretic model for competitive spectrum bidding and pricing. \textit{IEEE Wireless Communications}, 16(2), 16–23.
\bibitem{ref:ECMA} Standard ECMA-392. (2009). MAC and PHY for Operation in TV White Space.
\bibitem{ref:802.19.1} Sun, C., \& Tran, H.N., \& Rahman, M. A., \& Filin, S., \& Alemseged, Y. D., \& Villardi, G., \& Harada, H. (2009). P802.19.1 Assumptions and Architecture
\bibitem{ref:802.11} IEEE 802.11 Working Group on Wireless Local Area Networks, \url{http://www.ieee802.org/11/} Accessed 10 July 2013
\bibitem{ref:SCC} IEEE P1900.5 Policy Language and Policy Architectures for Managing Cognitive Radio for Dynamic Spectrum Access Applications \url{https:// ict-e3.eu/project/standardization/IEEE-SCC41.html} Accessed 10 July 2013
\bibitem{ref:Adler2012} Adler, J. (2012). Raging bulls: how Wall Street got addicted to light-speed trading. Wired.com. Retrieved from \url{http://www.wired.com/business/2012/08/ff_wallstreet_trading/all/} Accessed 15 June 2013

\bibitem{ref:Maharjan2011} Maharjan, S., Zhang, Y., \& Gjessing, S. (2011). Economic approaches for cognitive radio networks: a survey. \textit{Wireless Personal Communications}, 57(1), 33–51. doi:10.1007/s11277-010-0005-9
\bibitem{ref:Hossain2009} Hossain, E., Niyato, D., \& Han, Z. (2009). \textit{Dynamic spectrum access and management in cognitive radio networks}. Cambridge University Press.
\bibitem{ref:Niyato2008_Spec} Niyato, D., \& Hossain, E. (2008). Spectrum trading in cognitive radio networks: a market-equilibrium-based approach.\textit{ IEEE Wireless Communications}, (December), 71–80.
\bibitem{ref:Niyato2008_Mark} Niyato, D., \& Hossain, E. (2008). Market-equilibrium, competitive, and cooperative pricing for spectrum sharing in cognitive radio networks: analysis and comparison. \textit{IEEE Transactions on Wireless Communications}, 7(11), 4273–4283. 
\bibitem{ref:Niyato2007_Hier} Niyato, D., \& Hossain, E. (2007). Hierarchical spectrum sharing in cognitive radio: a microeconomic approach. \textit{IEEE Wireless Communications and Networking Conference, 2007.WCNC 2007.}, 3822–3826. doi:10.1109/WCNC.2007.699
\bibitem{ref:Niyato2007_Eq} Niyato, D., \& Hossain, E. (2007). Equilibrium and disequilibrium pricing for spectrum trading in cognitive radio: a control-theoretic approach. \textit{IEEE Global Telecommunications Conference, 2007. GLOBECOM  ’07.}, 4852–4856. doi:10.1109/GLOCOM.2007.920
\bibitem{ref:Niyato2010} Niyato, D., \& Hossain, E. (2010). A microeconomic model for hierarchical bandwidth sharing in dynamic spectrum access networks. \textit{IEEE Transactions on Computers}, 59(7), 865–877.
\bibitem{ref:Xu2011} Xu, P., Kapoor, S., \& Li, X. (2011). Market equilibria in spectrum trading with multi-regions and multi-channels. \textit{IEEE Global Telecommunications Conference (GLOBECOM 2011)}, 2011, 0–4.

\bibitem{ref:Niyato2007_Game} Niyato, D., \& Hossain, E. (2007). A game-theoretic approach to competitive spectrum sharing in cognitive radio networks. \textit{IEEE Wireless Communications and Networking Conference}, 2007.WCNC 2007., 16–20. doi:10.1109/WCNC.2007.9
\bibitem{ref:Mutlu2008} Mutlu, H., Alanyali, M., \& Starobinski, D. (2008). Spot pricing of secondary spectrum usage in wireless cellular networks. \textit{IEEE INFOCOM 2008. The 27th Conference on Computer Communications.}, 682–690. doi:10.1109/INFOCOM.2008.118
\bibitem{ref:Wang2008} Wang, F., Krunz, M., \& Cui, S. (2008). Price-based spectrum management in cognitive radio networks. \textit{IEEE Journal of Selected Topics in Signal Processing}, 2(1), 74–87. doi:10.1109/JSTSP.2007.914877
\bibitem{ref:Yu2010} Yu, H., Gao, L., Li, Z., Wang, X., \& Hossain, E. (2010). Pricing for uplink power control in cognitive radio networks. \textit{IEEE Transactions on Vehicular Technology}, 59(4), 1769–1778.
\bibitem{ref:Gao2011}  Gao, L., Wang, X., Xu, Y., \& Zhang, Q. (2011). Spectrum trading in cognitive radio networks : a contract-theoretic modeling approach. \textit{IEEE Journal on Selected Areas in Communications}, 29(4), 843–855.
\bibitem{ref:Yang2011} Yang, L., Kim, H., Zhang, J., Chiang, M., \& Tan, C. (2011). Pricing-based spectrum access control in cognitive radio networks with random access. \textit{2011 Proceedings IEEE INFOCOM}, 2228–2236.
\bibitem{ref:Xu2012} Xu, D., Liu, X., \& Han, Z. (2012). Decentralized bargain: a two-tier market for efficient and flexible dynamic spectrum access \textit{IEEE Transactions on Mobile Computing}, 1–1. doi:10.1109/TMC.2012.130

\bibitem{ref:Huang2006} Huang, J., Berry, R. A., \& Honig, M. L. (2006). Auction-based spectrum sharing. \textit{Mobile Networks and Applications}, 11(3), 405–418. doi:10.1007/s11036-006-5192-y
\bibitem{ref:Zhou2008} Zhou, X., Gandhi, S., Suri, S., \& Zheng, H. (2008). eBay in the Sky: strategy-proof wireless spectrum auctions. \textit{Proceedings of the 14th ACM international conference on Mobile computing and networking . MobiCom  ’08}, 2–13.  
\bibitem{ref:Huang2008_auc} Huang, J., Han, Z., Chiang, M., \& Poor, H. (2008). Auction-based resource allocation for cooperative communications. \textit{IEEE Journal on Selected Areas in Communications}, 26(7), 1226–1237. doi:10.1109/JSAC.2008.080919
\bibitem{ref:Wang2010_Spec} Wang, X., Li, Z., Xu, P., Xu, Y., Gao, X., \& Chen, H.-H. (2010). Spectrum sharing in cognitive radio networks--an auction-based approach. \textit{IEEE transactions on systems, man, and cybernetics—Part B, Cybernetics}, 40(3), 587–96. doi:10.1109/TSMCB.2009.2034630
\bibitem{ref:Gopinathan2011} Gopinathan, A., Li, Z., \& Wu, C. (2011). Strategyproof auctions for balancing social welfare and fairness in secondary spectrum markets. \textit{2011 Proceedings IEEE INFOCOM}, 3020–3028. doi:10.1109/INFCOM.2011.5935145
\bibitem{ref:Zhu2012} Zhu, Y., Li, B., \& Li, Z. (2012). Truthful spectrum auction design for secondary networks. \textit{2012 Proceedings IEEE INFOCOM}, 873–881. doi:10.1109/INFCOM.2012.6195836
\bibitem{ref:Gandhi2008} Gandhi, S., Buragohain, C., Cao, L., Zheng, H., \& Suri, S. (2008). Towards real-time dynamic spectrum auctions. \textit{Computer Networks}, 52(4), 879–897. doi:10.1016/j.comnet.2007.11.003

\bibitem{ref:Simeone2008} Simeone, O., Stanojev, I., Savazzi, S., Spagnolini, U., \& Pickholtz, R. (2008). Spectrum leasing to cooperating secondary ad hoc networks. \textit{IEEE Journal on Selected Areas in Communications}, 26(1), 203–213.
\bibitem{ref:Jayaweera2009} Jayaweera, S. K., \& Li, T. (2009). Dynamic spectrum leasing in cognitive radio networks via primary-secondary user power control games. \textit{IEEE Transactions on Wireless Communications}, 8(6), 3300–3310. doi:10.1109/TWC.2009.081230
\bibitem{ref:Zhang2009} Zhang, J., \& Zhang, Q. (2009). Stackelberg game for utility-based cooperative cognitive radio networks. \textit{Proceedings of the tenth ACM international symposium on Mobile ad hoc networking and computing. MobiHoc ’09}, 23–31.
\bibitem{ref:Jayaweera2010} Jayaweera, S. K., Vazquez-Vilar, G., \& Mosquera, C. (2010). Dynamic spectrum leasing : a new paradigm for spectrum sharing in cognitive radio networks. \textit{IEEE Transactions on Vehicular Technology}, 59(5), 2328–2339.  
\bibitem{ref:Vazquez2010} Vazquez-Vilar, G., Mosquera, C., \& Jayaweera, S. K. (2010). Primary user enters the game : performance of dynamic spectrum leasing in cognitive radio networks. \textit{IEEE Transactions on Wireless Communications}, 9(12), 3625–3629.
\bibitem{ref:Yi2010} Yi, Y., Zhang, J., Zhang, Q., Jiang, T., \& Zhang, J. (2010). Cooperative communication-aware spectrum leasing in cognitive radio networks. \textit{2010 IEEE Symposium on New Frontiers in Dynamic Spectrum}, 1–11. doi:10.1109/DYSPAN.2010.5457883
\bibitem{ref:Li2011} Li, D., Xu, Y., Wang, X., \& Guizani, M. (2011). Coalitional game theoretic approach for secondary spectrum access in cooperative cognitive radio networks. \textit{IEEE Transactions on Wireless Communications}, 10(3), 844–856. doi:10.1109/TWC.2011.011111.100216
\bibitem{ref:Duan2011_Contract} Duan, L., Gao, L., \& Huang, J. (2011). Contract-based cooperative spectrum sharing. 2011 IEEE International Symposium on Dynamic Spectrum Access Networks, 399–407.  

\bibitem{ref:Illeri2005} Ileri, O., Samardzija, D., \& Mandayam, N. B. (2005). Demand responsive pricing and competitive spectrum allocation via a spectrum server. \textit{2005 First IEEE International Symposium on New Frontiers in Dynamic Spectrum Access Networks, 2005. DySPAN 2005.}, 194–202.
\bibitem{ref:Xing2007} Xing, Y., Chandramouli, R., \& Cordeiro, C. (2007). Price dynamics in competitive agile spectrum access markets. \textit{IEEE Journal on Selected Areas in Communications}, 25(3), 613–621. doi:10.1109/JSAC.2007.070411
\bibitem{ref:Niyato2008_Comp} Niyato, D., \& Hossain, E. (2008). Competitive pricing for spectrum sharing in cognitive radio networks: dynamic game, inefficiency of Nash equilibrium, and collusion. \textit{IEEE Journal on Selected Areas in Communications}, 26(1), 192–202.
\bibitem{ref:Jia2008_com} Jia, J., \& Zhang, Q. (2008). Competitions and dynamics of duopoly wireless service providers in dynamic spectrum market. \textit{Proceedings of the 9th ACM international symposium on Mobile ad hoc networking and computing. MobiHoc  ’08}, 313–322.

\bibitem{ref:Duan2010_Cog} Duan, L., Huang, J., \& Shou, B. (2010). Cognitive mobile virtual network operator: investment and pricing with supply uncertainty. \textit{2010 Proceedings IEEE INFOCOM}, (February 2009).
\bibitem{ref:Duan2010_Comp} Duan, L., Huang, J., \& Shou, B. (2010). Competition with dynamic spectrum leasing. \textit{IEEE Symposium New Frontiers in Dynamic Spectrum Access Networks 2010}, 1–11.
\bibitem{ref:Duan2011_Duo} Duan, L., \& Huang, JianweiShou, B. (2011). Duopoly competition in dynamic spectrum leasing and pricing. \textit{IEEE Transactions On Mobile Computing}, 11(11), 1706–1719.
\bibitem{ref:Duan2011_Inves} Duan, L., Huang, J., \& Shou, B. (2011). Investment and pricing with spectrum uncertainty: a cognitive operator’s perspective. \textit{IEEE Transactions on Mobile Computing}, 10(11), 1590–1604.
\bibitem{ref:Zhu2012_Dyn} Zhu, K., Niyato, D., Wang, P., \& Han, Z. (2012). Dynamic spectrum leasing and service selection in spectrum secondary market of cognitive radio networks. \textit{IEEE Transactions on Wireless Communications}, 11(3), 1136–1145. doi:10.1109/TWC.2012.010312.110732
\bibitem{ref:Guijarro2011} Guijarro, L., Pla, V., Tuffin, B., Maille, P., \& Vidal, J. R. (2011). Competition and bargaining in wireless networks with spectrum leasing. \textit{2011 IEEE Global Telecommunications Conference (GLOBECOM 2011)}, 1–6.
\bibitem{ref:Min2011} Min, A., Zhang, X., Choi, J., \& Shin, K. G. (2012). Exploiting spectrum heterogeneity in dynamic spectrum market.\textit{ IEEE Transactions on Mobile Computing}, 11(12), 2020–2032.
\bibitem{ref:Kim2011} Kim, H., Choi, J., \& Shin, K. G. (2011). Wi-Fi 2.0: price and quality competitions of duopoly cognitive radio wireless service providers with time-varying spectrum availability. \textit{2011 Proceedings IEEE INFOCOM}, 2453–2461. doi:10.1109/INFCOM.2011.5935067
\bibitem{ref:Tan2010} Tan, Y., Sengupta, S., \& Subbalakshmi, K. P. (2010). Competitive spectrum trading in dynamic spectrum access markets: a price war. \textit{2010 IEEE Global Telecommunications Conference. GLOBECOM 2010}, 1–5. doi:10.1109/GLOCOM.2010.5683358
\bibitem{ref:Dixit2010} Dixit, S., Periyalwar, S., \& Yanikomeroglu, H. (2010). A competitive and dynamic pricing model for secondary users in infrastructure based networks. \textit{2010 IEEE 72nd Vehicular Technology Conference Fall (VTC 2010-Fall)}, 1–5. doi:10.1109/VETECF.2010.5594326
\bibitem{ref:Levi2012} Levent-Levi, T. (2012). Will MVNOs live long and prosper? Amdocs blogs. Retrieved from \url{http://blogs.amdocs.com/voices/2012/06/04/what-do-i-want-for-my-22nd-birthday-my-own-mvno-please/} Accessed 10 May 2013. 

\bibitem{ref:Wang2010_TODA} Wang, S., Xu, P., Xu, X., Tang, S., Li, X., \& Liu, X. (2010). TODA: Truthful Online Double Auction for spectrum allocation in wireless networks. \textit{2010 IEEE Symposium on New Frontiers in Dynamic Spectrum (DySPAN)}, 1–10. doi:10.1109/DYSPAN.2010.5457905
\bibitem{ref:Niyato2009_Dyn} Niyato, D., Hossain, E., \& Han, Z. (2009). Dynamics of multiple-seller and multiple-buyer spectrum trading in cognitive radio networks: a game-theoretic modeling approach. \textit{IEEE Transactions on Mobile Computing}, 8(8), 1009–1022.
\bibitem{ref:Gao2011_MAP} Gao, L., Xu, Y., \& Wang, X. (2011). MAP : Multiauctioneer Progressive auction for dynamic spectrum access. \textit{IEEE Transactions on Mobile Computing}, 10(8), 1144–1161. 
\bibitem{ref:Zhou2009_TRUST} Zhou, X., \& Zheng, H. (2009). TRUST : a general framework for truthful double spectrum auctions. \textit{IEEE INFOCOM 2009}, 999–1007.  
\bibitem{ref:Xu2010} Xu, H., Jin, J., \& Li, B. (2010). A secondary market for spectrum.\textit{ 2010 Proceedings IEEE INFOCOM}, 1–5. doi:10.1109/INFCOM.2010.5462277
\bibitem{ref:Jia2009_Rev} Jia, J., Zhang, Q., Zhang, Q., \& Liu, M. (2009). Revenue generation for truthful spectrum auction in dynamic spectrum access. In \textit{Proceedings of the tenth ACM international symposium on Mobile ad hoc networking and computing} (pp. 3–12). 

\bibitem{ref:Sengupta2007} Sengupta, S., Chatterjee, M., \& Ganguly, S. (2007). An economic framework for spectrum allocation and service pricing with competitive wireless service providers. \textit{2007 2nd IEEE International Symposium on New Frontiers in Dynamic Spectrum Access Networks}, 89–98. doi:10.1109/DYSPAN.2007.19
\bibitem{ref:Sengupta2009} Sengupta, S., \& Chatterjee, M. (2009). An economic framework for dynamic spectrum access and service pricing.\textit{ IEEE/ACM Transactions on Networking}, 17(4), 1200–1213. 
\bibitem{ref:Kaskebar2012} Kasbekar, G. S., \& Sarkar, S. (2010). Spectrum auction framework for access allocation in cognitive radio networks. \textit{IEEE/ACM Transactions on Networking}, 18(6), 1841–1854. doi:10.1109/TNET.2010.2051453 
\bibitem{ref:Ji2008} Ji, Z., \& Liu, K. J. R. (2008). Multi-stage pricing game for collusion-resistant dynamic spectrum allocation. \textit{IEEE Journal on Selected Areas in Communications}, 26(1), 182–191. doi:10.1109/JSAC.2008.080116

\bibitem{ref:Courcoubetis2003} Courcoubetis, C., \& Weber, R. (2003). \textit{Pricing communication networks: economics, technology and modelling}. Wiley.
\bibitem{ref:Yu2011_Cog} Yu, F. R. (2011). \textit{Cognitive radio mobile ad hoc networks}.  Springer.
\bibitem{ref:Liu2010} Liu, K., \& Wang, B. (2010). \textit{Cognitive radio networking and security: A game-theoretic view.} New York: Cambridge University Press.
\bibitem{ref:Hardin1968} Hardin, G. (1968). The Tragedy of the Commons. \textit{Science}, 162(3859), 1243–1248.
\bibitem{ref:MAC} Domenico, A. De, Strinati, E. C., \& Di Benedetto, M.-G. (2012). A survey on MAC strategies for cognitive radio networks. \textit{IEEE Communications Surveys Tutorials}, 14(1), 21–44.
\bibitem{ref:Han2007} Zhu, H., Pandana, C., \& Liu, K. J. R. (2007). Distributive opportunistic spectrum access for cognitive radio using correlated equilibrium and no-regret learning. \textit{IEEE Wireless Communications and Networking Conference, 2007.WCNC 2007.}, 11–15.
\bibitem{ref:Maskery2009} Maskery, M., Krishnamurthy, V., \& Zhao, Q. (2009). Decentralized dynamic spectrum access for cognitive radios: cooperative design of a non-cooperative game. \textit{IEEE Transactions on Communications}, 57(2), 459–469.

\bibitem{ref:Myerson1997} Myerson, R. B. (199). \textit{Game theory: analysis of conflict}. Harvard University Press. 
\bibitem{ref:Shen2013} Shen, S., Lin, X., \& Lok, T. M. (2013). Dynamic Spectrum Leasing under uncertainty: A stochastic variational inequality approach. \textit{2013 IEEE Wireless Communications and Networking Conference (WCNC)}, 727–732. doi:10.1109/WCNC.2013.6554653
\bibitem{ref:Zhang2012} Zhang, Y., Niyato, D., Wang, P., \& Hossain, E. (2012). Auction-based resource allocation in cognitive radio systems. IEEE Communications Magazine, 50(11), 108–120. doi:10.1109/MCOM.2012.6353690
\bibitem{ref:Yan2012} Yan, Y., Huang, J., \& Wang, J. (2012). Dynamic bargaining for relay-based cooperative spectrum sharing. \textit{IEEE Journal on Selected Areas in Communications}, (August), 1480–1493.
\bibitem{ref:Zhang2012_Fair}. Zhang, G., Yang, K., Song, J., \& Li, Y. (2012). Fair and efficient spectrum splitting for unlicensed secondary users in cooperative cognitive radio networks. \textit{Wireless Personal Communications}, 71(1), 299–316. doi:10.1007/s11277-012-0816-y
\bibitem{ref:Pan2006} Pan, M., Liang, S., Xiong, H., Chen, J., \& Li, G. (2006). A novel bargaining based dynamic spectrum management scheme in reconfigurable systems. In \textit{2006 International Conference on Systems and Networks Communications (ICSNC’06)} (Vol. 00, pp. 54–54). IEEE. doi:10.1109/ICSNC.2006.10
\bibitem{ref:Ji2006} Ji, Z., \& Liu, K. J. R. (2006). WSN03-3: Dynamic pricing approach for spectrum allocation in wireless networks with selfish users. In \textit{IEEE Globecom 2006} (pp. 1–5). IEEE. doi:10.1109/GLOCOM.2006.939
\bibitem{ref:Gao2013} Gao, L., Huang, J., \& Shou B. (2013). An integrated contract and auction design for secondary spectrum trading. \textit{IEEE Journal on Selected Areas in Communications}, (March), 581–592. 
\bibitem{ref:Duan2014} Duan, L., Gao, L., \& Huang, J. (2014). Cooperative spectrum sharing: a contract-based approach. \textit{IEEE Transactions on Mobile Computing}, 13(1), 174–187. 
\bibitem{ref:Vidal2013} Vidal, J. R., Pla, V., Guijarro, L., \& Martinez-Bauset, J. (2013). Dynamic spectrum sharing in cognitive radio networks using truthful mechanisms and virtual currency. \textit{Ad Hoc Networks}, 11(6), 1858–1873. doi:10.1016/j.adhoc.2013.04.010
\bibitem{ref:Kim2013} Kim, S. (2013). A repeated Bayesian auction game for cognitive radio spectrum sharing scheme. \textit{Computer Communications}, 36(8), 939–946. doi:10.1016/j.comcom.2013.02.003
\bibitem{ref:Lee2011}Lee, K., Simeone, O., Chae, C.-B., \& Kang, J. (2011). Spectrum leasing via cooperation for enhanced physical-layer secrecy. In \textit{2011 IEEE International Conference on Communications (ICC)} (pp. 1–5). IEEE. doi:10.1109/icc.2011.5963501
\bibitem{ref:Elias2013} Elias, J., \& Martignon, F. (2013). Joint operator pricing and network selection game in cognitive radio networks: equilibrium, system dynamics and price of anarchy. \textit{IEEE Transactions on Vehicular Technology}, 62(9), 4576–4589. 
\bibitem{ref:Galla2013} Galla, T., \& Farmer, J. D. (2013). Complex dynamics in learning complicated games. \textit{Proceedings of the National Academy of Sciences of the United States of America}, 110(4), 1232–6. doi:10.1073/pnas.1109672110
\bibitem{ref:Yan2011} Yan, Y., Huang, J., Zhong, X., Zhao, M., \& Wang, J. (2011). Sequential bargaining in cooperative spectrum sharing: incomplete information with reputation effect. In \textit{2011 IEEE Global Telecommunications Conference - GLOBECOM 2011} (pp. 1–5). IEEE. doi:10.1109/GLOCOM.2011.6134516
\bibitem{ref:Pantisano} F. Pantisano, \textit{et. al.}, ``Spectrum leasing as an incentive towards uplink macrocell and femtocell cooperation,'' \textit{IEEE J. Select. Areas Commun.}, vol.30, no.3, pp.617-30, Apr. 2012.
\bibitem{ref:Yi} Yi, Y., Zhang, J., Zhang, Q., Jiang, T., \& Zhang, J. (2010), ``Cooperative communication-aware spectrum leasing in cognitive radio networks,'' \textit{2010 IEEE Symposium on New Frontiers in Dynamic Spectrum}, Apr. 2010.
\bibitem{ref:Huang2013} Huang, J. (2013). Market mechanisms for cooperative spectrum trading with incomplete network information. \textit{IEEE Communications Magazine}, (October), 201–207. 
\bibitem{ref:Zhang2013} Zhang, Z., Long, K., \& Wang, J. (2013). Self-organization paradigms and optimization approaches for cognitive radio technologies: a survey. \textit{IEEE Wireless Communications}, (April), 36–42. 
\bibitem{ref:Akkara2011} Akkarajitsakul, K., Hossain, E., Niyato, D., \& Kim, D. I. (2011). Game theoretic approaches for multiple access in wireless networks: a survey. \textit{IEEE Communications Surveys \& Tutorials}, 13(3), 372–395. doi:10.1109/SURV.2011.122310.000119
\bibitem{ref:Shoham2007}1. Shoham, Y., Powers, R., \& Grenager, T. (2007). If multi-agent learning is the answer, what is the question? \textit{Artificial Intelligence}, 171(7), 365–377. doi:10.1016/j.artint.2006.02.006





% Sarnoff

\bibitem{ref:110}
FCC Spectrum policy Task Force ``Report on the spectrum efficiency group,'' FCC Report, Nov. 2002.


\bibitem{ref:Puterman} M. L. Puterman ``Markov Decision Processes: Discrete Stochastic Dynamic Programming,'' First Edition  \emph{Wiley-Interscience}, 2005.

\bibitem{ref:Powell} W. B. Powell ``Approximate Dynamic Programming: Solving the Curses of Dimensionality,'' First Edition  \emph{Wiley-Interscience}, 2007.

%\bibitem{ref:Zhao2007} Q. Zhao, L. Tong, A. Swami and Y. Chen, ``Decentralized cognitive MAC for opportunistic spectrum access in ad hoc networks: a POMDP framework,'' \emph{IEEE Journal on Selected Areas in Communications}, vol. 25, no. 3, pp. 589-600, April 2007.

%\bibitem{ref:Geirhofer2008} S. Geirhofer, L. Tong and B.M. Sadler, ``Cognitive Medium Access: Constraining Interference Based on Experimental Models,'' \emph{Selected Areas in Communications, IEEE Journal on }, vol.26, no.1, pp.95-105, Jan. 2008.

%\bibitem{ref:Chen2008} Y. Chen, Q. Zhao and A. Swami, ``Joint Design and Separation Principle for Opportunistic Spectrum Access in the Presence of Sensing Errors,'' \emph{Information Theory, IEEE Transactions on }, vol.54, no.5, pp.2053-2071, May 2008.

%\bibitem{ref:Li2011} X. Li, Q. Zhao, X. Guan and L. Tong, ``Optimal Cognitive Access of Markovian Channels under Tight Collision Constraints,'' \emph{Selected Areas in Communications, IEEE Journal on}, vol.29, no.4, pp.746-756, April 2011.

\bibitem{ref:Yu2007} O. Yu, E. Saric and A. Li, ``Dynamic control of open spectrum management,'' in \emph{Proceedings of IEEE Wireless Communications and Networking Conference} (WCNC), March
2007, pp. 127-132.

%\bibitem{ref:Hoang2010} A. Hoang, Y.-C. Liang and Y. Zeng., ``Adaptive joint scheduling of spectrum sensing and data transmission in cognitive radio networks,'' \emph{Communications, IEEE Transactions on }, vol.58, no.1, pp.235-246, Jan. 2010.

%\bibitem{ref:Berthold2008} U. Berthold,F. Fangwen, M. van der Schaar and F.K. Jondral, ``Detection of Spectral Resources in Cognitive Radios Using Reinforcement Learning,'' in Proc. \emph{New Frontiers in Dynamic Spectrum Access Networks, 2008. DySPAN 2008. 3rd IEEE Symposium on }, vol., no., pp.1-5, 14-17 Oct. 2008.

%\bibitem{ref:Galindo2010} A. Galindo-Serrano and L. Giupponi, ``Distributed Q-Learning for Aggregated Interference Control in Cognitive Radio Networks,'' \emph{Vehicular Technology, IEEE Transactions on }, vol.59, no.4, pp.1823-1834, May 2010.

\bibitem{ref:Niyato} D. Niyato, E. Hossain and Z. Han, ``Dynamics of Multiple-Seller and Multiple-Buyer Spectrum Trading in Cognitive Radio Networks: A Game-Theoretic Modeling Approach,'' \emph{ IEEE Transactions on Mobile Computing}, vol.8, no.8, pp.1009-22, Aug. 2009.

\bibitem{ref:eBay} X. Zhou, S. Gandhi, S. Suri, and H. Zheng, ``eBay in the Sky: Strategy- Proof Wireless Spectrum Auctions,'' in \emph{Proc. of ACM MobiCom}, 2008.

\bibitem{ref:Jia} J. Jia, Q. Zhang , Q. Zhang and M. Liu, ``Revenue Generation for Truthful Spectrum Auction in Dynamic Spectrum Access,'' \emph{Proc. of the ACM MobiHoc'09}, May 18-21, 2009.

\bibitem{ref:Wang} X. Wang, Z. Li, P. Xu, Y. Xu, X. Gao, and H.-H. Chen, ``Spectrum Sharing in Cognitive Radio Networks - an Auction Based Approach,'' \emph{IEEE Transactions on Systems, Man, and Cybernetics, Part B: Cybernetics}, vol. 40, no. 3, pp. 587-596, 2010.

\bibitem{ref:Klei} L. Kleinrock, ``Queuing Systems, Volume 1: Theory,'' \emph{John Wiley \& Sons}, New York, 1975.























% MAB_CSSA
\bibitem{ref:Jorswieck2013}
E.~A. Jorswieck, L.~Badia, T.~Fahldieck, E.~Karipidis, and J.~Luo, ``Spectrum
  sharing improves the network efficiency for cellular operators,''
  \emph{{IEEE} Commun. Mag.}, vol.~52, no.~3, pp. 129--136, Dec. 2013.

\bibitem{ref:Mario2015}
M.~L\'{o}pez-Mart\'{\i}nez, J.~J. Alcaraz, L.~Badia, and M.~Zorzi,
  ``Multi-armed bandits with dependent arms for cooperative spectrum sharing,''
  in \emph{{IEEE} International Conference on Communications {ICC}}, 2015, to
  be published.

\bibitem{ref:Auer2002}
P.~Auer, N.~Cesa-Bianchi, and P.~Fischer, ``Finite-time analysis of the
  multiarmed bandit problem,'' \emph{Machine learning}, vol.~47, no. 2-3, pp.
  235--256, 2002.

\bibitem{ref:Gittins2011}
J.~Gittins, K.~Glazebrook, and R.~Weber, \emph{Multi-armed Bandit Allocation
  Indices}.\hskip 1em plus 0.5em minus 0.4em\relax West Sussex, UK: Wiley,
  2011.

\bibitem{ref:Feng2014}
X.~Feng, G.~Sun, X.~Gan, F.~Yang, and X.~Tian, ``Cooperative spectrum sharing
  in cognitive radio networks: A distributed matching approach,'' \emph{{IEEE}
  Trans. Commun.}, vol.~62, no.~8, pp. 2651--2644, Aug. 2014.


\bibitem{ref:Yan2013}
Y.~Yan, J.~Huang, and J.~Wang, ``Dynamic bargaining for relay-based cooperative
  spectrum sharing,'' \emph{{IEEE} J. Sel. Areas Commun.}, vol.~31, no.~8, pp.
  1480--1493, Aug. 2013.

\bibitem{ref:Yuan2013}
X.~Yuan, Y.~Shi, Y.~T. Hou, W.~Lou, and S.~Kompella, ``{UPS}: A united
  cooperative paradigm for primary and secondary networks,'' in \emph{{IEEE}
  10th International Conference on Mobile Ad-Hoc and Sensor Systems}, 2013, pp.
  78--85.

\bibitem{ref:Nadkar2011}
T.~Nadkar, V.~Thumar, G.~Shenoy, A.~Mehta, U.~B. Desai, and S.~N. Merchant, ``A
  cross-layer framework for symbiotic relaying in cognitive radio networks,''
  in \emph{{IEEE} International Symposium on Dynamic Spectrum Access Networks
  (DySPAN)}, May 2011, pp. 498--509.

\bibitem{ref:Han2010}
Y.~Han, S.~H. Ting, and A.~Pandharipande, \emph{{IEEE} Trans. Wireless
  Commun.}, vol.~9, no.~9, pp. 2914--2923, Sep. 2010.


\bibitem{ref:Niyato2008}
D.~Niyato and E.~Hossain, ``Market-equilibrium, competitive, and cooperative
  pricing for spectrum sharing in cognitive radio networks: Analysis and
  comparison,'' \emph{{IEEE} Trans. Wireless Commun.}, vol.~7, no.~11, pp.
  4273--4283, Nov. 2008.

\bibitem{ref:Jayaweera2011}
S.~Jayaweera, M.~Bkassiny, and K.~A. Avery, ``Asymmetric cooperative
  communications based spectrum leasing via auctions in cognitive radio
  networks,'' \emph{{IEEE} Trans. Wireless Commun.}, vol.~10, no.~8, pp.
  2716--2724, 2011.

\bibitem{ref:Alcaraz2014_coa}
J.~Alcaraz and M.~van~der Schaar, ``Coalitional games with intervention:
  Application to spectrum leasing in cognitive radio,'' \emph{{IEEE} Trans.
  Wireless Commun.}, vol.~PP, no.~99, pp. 1--1, 2014.

\bibitem{ref:Calvo2002}
A.~Calvo-Armengol, ``On bargaining partner selection when communication is
  restricted,'' \emph{International Journal of Game Theory}, vol.~30, no.~4,
  pp. 503--515, Jan. 2002.

\bibitem{ref:Tran2014}
T.~T. Tran and H.~Y. Kong, ``Exploitation of diversity in cooperative spectrum
  sharing with the four-way relaying af transmission,'' \emph{Wireless Personal
  Communications}, vol.~77, no.~4, pp. 2959--2980, Aug. 2014.

\bibitem{ref:Brown2013}
D.~B. Brown and J.~E. Smith, ``{Optimal Sequential Exploration: Bandits,
  Clairvoyants, and Wildcats},'' \emph{Operations Research}, vol.~61, no.~3,
  pp. 644--665, Jun. 2013.

\bibitem{ref:Pandey2007}
S.~Pandey, D.~Chakrabarti, and D.~Agarwal, ``Multi-armed bandit problems with
  dependent arms,'' in \emph{{ICML} Proceedings of the 24th International
  Conference on Machine learning}, 2007, pp. 721--728.

\bibitem{ref:Si2010}
P.~Si, H.~Ji, F.~R. Yu, and V.~C.~M. Leung, ``Optimal cooperative internetwork
  spectrum sharing for cognitive radio systems with spectrum pooling,''
  \emph{{IEEE} Trans. Veh. Technol.}, vol.~59, no.~4, pp. 1760--1768, May 2010.

\bibitem{ref:Gavrilovska2013}
L.~Gavrilovska, V.~Atanasovski, I.~Macaluso, and L.~A. Dasilva, ``Learning and
  reasoning in cognitive radio networks,'' \emph{{IEEE} Commun. Surveys Tuts.},
  vol.~15, no.~4, pp. 1761--1777, Mar. 2013.

\bibitem{ref:Shao2014}
C.~Shao, H.~Roh, and W.~Lee, ``Aspiration level-based strategy dynamics on the
  coexistence of spectrum cooperation and leasing,'' \emph{{IEEE}
  Communications Letters}, vol.~18, no.~1, pp. 70--73, Jan. 2014.

\bibitem{ref:Goldsmith2009}
A.~Goldsmith, S.~{Jafar Ali}, I.~Maric, and S.~Srinivasa, ``Breaking spectrum
  gridlock with cognitive radios : An information theoretic perspective,''
  \emph{Proc. {IEEE}}, vol.~97, no.~5, pp. 894 -- 914, May 2009.

\bibitem{ref:Laneman2001}
J.~N. Laneman and G.~W. Wornell, ``An efficient protocol for realizing
  distributed spatial diversity in wireless ad-hoc networks,'' in \emph{Proc.
  of ARL FedLab Symposium on Advanced Telecommunications and Information
  Distribution}, Washington, DC, 2001, pp. 294--.

\bibitem{ref:Bertsekas2005}
D.~Bertsekas, \emph{Dynamic Programming and Optimal Control}.\hskip 1em plus
  0.5em minus 0.4em\relax Nashua, NH: Athena Scientific, 2000.

\bibitem{ref:Whittle1980}
P.~Whittle, ``Multi-armed bandits and the gittins index,'' \emph{Journal of the
  Royal Statistical Society. Series {B} (Methodological)}, vol.~42, no.~2, pp.
  143--149, 1980.

\bibitem{ref:Gass1955}
S.~Gass and T.~Saaty, ``The computational algorithm for the parametric
  objective function,'' \emph{Naval Research Logistics}, vol.~2, no. 1-2, pp.
  39--45, Mar. 1955.

\bibitem{ref:Vermorel2005}
J.~Vermorel and M.~Mohri, ``Multi-armed bandit algorithms and empirical
  evaluation,'' in \emph{Machine Learning: {ECML}}, 2005, pp. 437--448.

\bibitem{ref:Goldsmith2005}
A.~Goldsmith, \emph{Wireless Communications}.\hskip 1em plus 0.5em minus
  0.4em\relax Cambridge University Press, 2005


%Conclusion
\bibitem{ref:RSPG2009} ERG-RSPG Report on radio spectrum competition issues, ERG (09) 22, RSPG09-279 Rev.2, at page 32 (June 2009)
\bibitem{ref:Damnjanovic2011} Damnjanovic, A.; Montojo, J.; Yongbin Wei; Tingfang Ji; Tao Luo; Vajapeyam, M.; Taesang Yoo; Osok Song; Malladi, D., \enquote{A survey on 3GPP heterogeneous networks,} \textit{IEEE Wireless Communications}, vol.18, no.3, pp.10,21, June 2011
\bibitem{ref:Andrews2014} J. G. Andrews, S. Buzzi, W. Choi, S. V Hanly, A. Lozano, a C. K. Soong, and J. C. Zhang, \enquote{What Will 5G Be?,} \textit{IEE J. Sel. Areas Commun-}, vol. 32, no. 6, pp. 1065–1082, 2014.
\end{thebibliography}